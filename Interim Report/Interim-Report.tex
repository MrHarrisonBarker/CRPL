\documentclass[12pt]{report}

\usepackage[english]{babel}

\usepackage{graphicx}
\usepackage[colorlinks=true, allcolors=blue]{hyperref}

\bibliographystyle{plain}

\title{CRPL: Interim Report}
\author{Harrison Beau Barker}

\newcommand{\keyword}[1]{\textbf{\textit{#1}}}
\newcommand{\q}[2]{\begin{quote} #1 \cite{#2} \end{quote}}

\setlength{\parskip}{1em}
\setlength{\parindent}{0em}
\begin{document}

\maketitle
\abstract{This is the abstract}

\tableofcontents{}

\chapter{Keywords}

\begin{description}
  \item[Blockchain] A blockchain is essentially an immutable data structure made of hashed \keyword{blocks} not dissimilar to a linked list. Each block contains a timestamp, the actual data stored, and a hash (Ethereum uses Keccak-256 for hashing) of the previous block. this is what makes this data structure immutable and more importantly secure because once a block has been hashed you can't go back and change it otherwise the hash would be different, this also means the more blocks hashed on the chain the harder it is to change the past as new blocks depend on the validity of the previous.
  \item[Ethereum] Ethereum simply is an open-source blockchain with its own cryptocurrency Ether, however its real claim to fame is its \keyword{smart contract} functionality thanks to the Ethereum Virtual Machine (EVM) which allows for the existence of powerful concepts most importantly \keyword{state} and to be able to modify the current state. Ether is currently the second most valuable crypto currency in the world and the ethereum blockchain is the largest of its kind.
  \item[Smart Contract] As talked about previously smart contracts are pieces of software stored within the Ethereum network, these programs have an address, Ether balance, and can transact. All smart contracts are "controlled" via transaction which of course are saved to the blockchain this interaction is therefore irreversible which allows for the changing of sate and the history of state to forever be recorded.
  \item[Intellectual property] Is a blanket term describing any property created by the human intellect specifically intangible property eg; a carpenter who designs a new type of chair is only protected in terms of IP for the design or any unique building practices but not the physical chair itself which would be subject to other property laws.
  \item[Copyright] Is therefore a specific type of Intellectual property law which pertains to the copying and distribution of a work, copyright is extremely important for the protection of works as it ratifies who can essentially make money from the work.
\end{description}

\chapter{Introduction}

CRPL or \textbf{C}opy\textbf{r}ight on a \textbf{p}ublic \textbf{l}edger is an open platform for registering and managing the copyrights of a creators intellectual property secured on the publicly distributed ledger Ethereum. The immutability and public distribution of a blockchain is the essential component to the success of this platform and allows copyright registration to be placed in everyones hands.

The blockchain industry is currently booming with thousands of startups and big name players getting involved innovating with this new technology, however I believe there is a big misstep in the application of blockchain technology as generally the systems and product being made are using its as a marketing pitch not a fundamentally different way of creating a system. Most are web apps we already have but with blockchain internals whereas the application of blockchain within social and democratic issues is immense (Don and Alex Tapscott talk about this topic in more depth in their book Blockchain Revolution\cite{blockchain_revolution} which in part inspired me to develop this particular project) and should be the real focus of this new technology.

The core motivation for this project is two fold: first is the simplification or disintermediation of intellectual property rights and secondly is the decentralisation of copyright. The idea of copyright is simple however the implementation is far from, a large part of this comes from the fact that copyright existed before the internet meaning governments had to each implement their own idea of copyright this became a real problem when intellectual property went global thanks to the internet. Now each system has to sort of mesh with each others which can be extremely complex especially when it comes to smaller independent creators.

I believe my solution will solve a lot of these issues by making the protection of intellectual property easy and open.

\chapter{Background/Research}

\section{Blockchain}

The theory of a blockchain is not new \cite{origins_blockchain} but the implementation and hype around blockchains is new and extremely popular in the current day thanks to Bitcoin designed by Satoshi Nakamoto based on their white paper \cite{nakamoto2008bitcoin}. The main tenants of a blockchain are decentralisation, distribution, and immutability. Therefore for a blockchain to exist and be useful it needs people or nodes to maintain the data structure authenticate all transactions in a block and hash the current block in preparation for the next in the chain.

%TODO graphviz diagram of a blockchain?

\subsection{Technical proficiencies / limitations}

In this section I will be discussing the technical pros and cons of a blockchain system

\subsubsection{Proficiencies}

The three main selling points of blockchain technology are as follows:

\begin{enumerate}
	\item \textbf{Security} Is the core tenant of a blockchain as the cryptographic hashing strategy is what makes the data structure a chain.
	\item \textbf{Transparency} Is not intrinsic to blockchains that is down to the implementation ie: if it's public or not, however the underlying concept facilitates and promotes this open and transparent behaviour simply by the fact that nodes within a blockchain network need a copy of all transactions/previous blocks.
	\item \textbf{Decentralisation} Is definitely the most talked about benefit of blockchains and it's not a technical benefit just like transparency decentralisation is a social or philosophical debate that is made possible by the technical decisions of the blockchain architecture, mostly centred around the control and ownership of peoples data \begin{quote}"It avoids concentrations of power that could let a single person or organization take control."\cite{bohme2015bitcoin}\end{quote} 
\end{enumerate}

% blockchain in government \cite{OLNES2017355}
% trust \cite{7163223}

\subsubsection{Limitations}

Blockchain technology faces two major limitations, the largest chains in the world (Bitcoin and Ethereum) are quickly growing in size as they progress through the adoption phase of their lifecycles which has clearly pointed out scaleability problems relating to transaction speed and cost per transaction which are currently slow and expensive respectively compared to their traditional counterparts (cost experiments have borne this out in the aptly named paper "Comparing blockchain and cloud services for business process execution"\cite{rimba2017comparing} and show that business logic on a blockchain is 2 orders of magnitude more expensive than current cloud services).

Blockchains rely on \keyword{consensus protocols} which is a mathematical formula or process to determine consensus of the whole network, simply enough of the nodes/network has to agree the current state of the blockchain whenever it's modified. Currently all major chains including Bitcoin and Ethereum use a protocol called proof of work \cite{PoW} which uses the total amount of compute a node has produced as the comparable proof that can be verified by many other nodes easily. This has worked so far for these major chains however ballooning energy consumption (which can be seen in realtime at the \href{https://ccaf.io/cbeci/index}{Cambridge Bitcoin Electricity Consumption Index}), high transaction costs, and inadequate transaction processing bandwidth is forcing them to change and look for solutions.

% TODO pad this out a bit
Ethereum have decided to completely reinvent their consensus protocol with a fundamentally different protocol called proof of stake which promises reduced energy consumption, cheaper, and faster transactions \cite{PoS}

Although there have been other proposed solutions including: improving the proof of stake system by parallelising it across nodes \cite{fi12080125} or reducing the block size for a increase transaction capacity by reducing the amount of work needed per block however there will be a security trade off to balance \cite{kiayias2015speed}.

\subsection{Use of blockchain for IP protection}

I am not the first person to have thought of representing and enforcing copyright protection on a distributed blockchain, there has actually been a lot of discussion around the topic showing the clear benefits of using this technology for copyright such as transparency; \begin{quote} "Blockchain may substantially increase visibility and availability of information about copyright ownership." \cite{Copyright_in_the_blockchain_era} \end{quote} the power of smart contracts and utilising a networks cryptocurrency \begin{quote} "Smart contracts will allow automatic and instantaneous payments to designated parties, and expiration of a license after a certain amount of time." \cite{Copyright_in_the_blockchain_era} \end{quote} and simplification through the globalisation of copyright law.

Overall consensus is the addition of \keyword{smart contracts} to blockchain technology is extremely powerful particularly within copyright law to \begin{quote}"reliably automate a large volume of ‘dumb transactions’" \cite{missing_link_in_copyright_licensing}\end{quote} which will greatly help to reduce friction by removing unnecessary work and removing the expertise needed currently in the field to be properly protected.

\subsection{Use of blockchain for legal purposes}

The discussion around bringing copyright to blockchain is really apart of a larger conversation about the compatibility of any legal or even governmental workflows to be either represented or completely reinvented using blockchain technology, and it does look like that in many cases these types of problems can leverage blockchain and possibly even thrive.

First I'll look at the legal industry which is notoriously complicated requiring a great level of knowledge and expertise in the field to make sense of anything or more importantly get anything done. Logically laws make an obvious starting point for computerisation because computers are defined systems of ridged laws however this didn't make sense during the first wave of blockchains namely Bitcoin which almost entirely focused its efforts towards currency. Where as the introduction of \keyword{smart contracts} has opened up the possibilities for automated law processing massively reducing boilerplate and bulky legal work which is largely trivial but time consuming.

\begin{quote}
	"So-called ‘smart contracts’ built on blockchain technologies may prove to be the most important example yet of “self-executing, customised rules”." \cite{MILLARD2018843}
\end{quote}

Within a governmental setting blockchains \keyword{ledger} and transparency will do all the shining as essentially all governments are big collections of "things", assets, people and information. Not only are they collections but historical records and public institutions, thankfully blockchains are immutable, timestamped, secure and built for openness. Because a blockchain is just a general purpose data structure its uses are wide; \begin{quote}"keeping an overview of the authorities provided in a public organization and the ability to change the authority only if there is agreement among nodes which are classified as being higher ranked in the hierarchy." \cite{OLNES2017355}\end{quote}

\section{Copyright law}
%TODO something to show that copyright needs a change 

What is the reason for "fixing" copyright? is it broken? Copyright law is old, it was first introduced from the 1700s to the 1800s with the introduction of the printing press and the first true copyright act being the Statute of Anne in 1710 which has morphed into the Copyright, Designs and Patents Act 1988 in the UK (link!!!!!!!) and the Copyright Act of 1976 in the US. The fact that each country has its own unique copyright laws (even though the international Berne Convention exists local jurisdictions still apply) exacerbated by the internet has been showing signs of cracks especially when "digital goods" are involved.

Theses issues can be broken down into three key areas: 

\begin{enumerate}
	\item \textbf{Legal transparency/complexity} Is the absolutely at the top of this list, as a creative can you be sure your work is protected in every jurisdiction possible? No you can't because the current copyrighting system is a sparse disconnected collection of closed databases. Implementing a globally open database of copyrighted works with consistent rules and digital finger printing. 
	\item \textbf{Royalty distributions (compensation)} Monetary transactions are the current forte of all major blockchains so why not simplify one of the hardest tasks for an creator, receiving payment for the use of a work of course this has become easier with the introduction of internet payments and purpose built systems like "PRS for Music" but integrating royalty payments straight into the copyright registration proves for secure and cohesive system.
	\item \textbf{Cost} Of protection a work is essentially free as the Berne Convention prohibits the registration of a work as a requirement to protection (copyright is born with the work), however legally protecting and defending authorship of a work is not free and can be devastating not to mention extremely one sided (massive record company vs band). With the help of an immutable global ledger determining who registered the work is simple. % smart contract laws??
\end{enumerate}

These problems and solutions in this section are talked about in the paper \textit{Copyright in the blockchain era: Promises and challenges} \cite{Copyright_in_the_blockchain_era}

%\section{Environment and crypto} not enough for this section

\section{The market}

%TODO probably a bit more ?
The vast majority of blockchain applications in the current day are financial for obvious reasons with the market capitalisation of cryptocurrencies being the universal metric for the blockchain markets size \cite{wood_2021}, however the health care industry has been aggressively investigating and implementing blockchain technology particularly in the secure distribution of medical data and health records \cite{8167115}, supply chain technology has seen some innovation tracking goods as they pass through a chain. 

\chapter{Methods/Technology used}
%TODO methods used to develop the project including technical solutions

\section{C\#}
\section{ASP.NET}
\section{Angular}
\section{Nethereum}
\section{Solidity}
\section{Sql vs NoSql}
\section{Wallets}

\section{Methodologies}
\subsection{Agile}
\subsection{TDD}

\section{Tools}
\subsection{Geth}
\subsection{Azure}

\chapter{Results}
%TODO user feedback (supervisor)

\chapter{Analysis}
%TODO has everything done so far backed up what was originally planned

\chapter{Conclusion}
%TODO main takeaways so far

\chapter{Progress and the future}
%TODO what have i done so far (prototype, task backlog, design feedback), what needs to get done for the project to be finished

% Have i needed to change the direction of the project?
% Will I finish on time

\section{Backlog}
%TODO What does my backlog look like



\bibliography{sources.bib}

\end{document}
