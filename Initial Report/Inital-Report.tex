\documentclass[12pt]{report}

\usepackage[english]{babel}

\usepackage{graphicx}
\usepackage[colorlinks=true, allcolors=blue]{hyperref}

\bibliographystyle{plain}

\title{CRPL: Initial Report}
\author{Harrison Beau Barker}

\setlength{\parskip}{1em}
\begin{document}

\maketitle
\abstract{This is the abstract}

\tableofcontents{}

\chapter{Keywords}

\begin{description}
  \item[Blockchain] A blockchain is essentially an immutable data structure made of hashed \textit{blocks} not dissimilar to a linked list. Each block contains a timestamp, the actual data stored, and a hash (Ethereum uses Keccak-256 for hashing) of the previous block. this is what makes this data structure immutable and more importantly secure because once a block has been hashed you can't go back and change it otherwise the hash would be different, this also means the more blocks hashed on the chain the harder it is to change the past as new blocks depend on the validity of the previous.
  \item[Ethereum] Ethereum simply is an open-source blockchain with its own cryptocurrency Ether, however its real claim to fame is its \textit{smart contract} functionality thanks to the Ethereum Virtual Machine (EVM) which allows for the existence of powerful concepts most importantly \textbf{state} and to be able to modify the current state. Ether is currently the second most valuable crypto currency in the world and the ethereum blockchain is the largest of its kind.
  \item[Smart Contract] As talked about previously smart contracts are pieces of software stored within the Ethereum network, these programs have an address, Ether balance, and can transact. All smart contracts are "controlled" via transaction which of course are saved to the blockchain this interaction is therefore irreversible which allows for the changing of sate and the history of state to forever be recorded.
  \item[Intellectual property] Is a blanket term describing any property created by the human intellect specifically intangible property eg; a carpenter who designs a new type of chair is only protected in terms of IP for the design or any unique building practices but not the physical chair itself which would be subject to other property laws.
  \item[Copyright] Is therefore a specific type of Intellectual property law which pertains to the copying and distribution of a work, copyright is extremely important for the protection of works as it ratifies who can essentially make money from the work.
\end{description}

\chapter{Introduction}

CRPL or \textbf{C}opy\textbf{r}ight on a \textbf{p}ublic \textbf{l}edger is an open platform for registering and managing the copyrights of a creators intellectual property secured on the publicly distributed ledger Ethereum. The immutability and public distribution of a blockchain is the essential component to the success of this platform and allows copyright registration to be placed in everyones hands.

The blockchain industry is currently booming with thousands of startups and big name players getting involved innovating with this new technology, however I believe there is a big misstep in the application of blockchain technology as generally the systems and product being made are using its as a marketing pitch not a fundamentally different way of creating a system. Most are web apps we already have but with blockchain internals whereas the application of blockchain within social and democratic issues is immense (Don and Alex Tapscott talk about this topic in more depth in their book Blockchain Revolution\cite{blockchain_revolution} which in part inspired me to develop this particular project) and should be the real focus of this new technology.

The core motivation for this project is two fold: first is the simplification or disintermediation of intellectual property rights and secondly is the decentralisation of copyright. The idea of copyright is simple however the implementation is far from, a large part of this comes from the fact that copyright existed before the internet meaning governments had to each implement their own idea of copyright this became a real problem when intellectual property went global thanks to the internet. Now each system has to sort of mesh with each others which can be extremely complex especially when it comes to smaller independent creators.

I believe my solution will solve a lot of these issues by making the protection of intellectual property easy and open.

\chapter{Background/Research}
%TODO show that I've done all the reading/research needed to create the project (crypto, blockchain, existing copyright law!!, agile dev, TDD)

\chapter{Methods/Technology used}
%TODO methods used to develop the project including technical solutions

\chapter{Results}
%TODO user feedback (supervisor)

\chapter{Analysis}
%TODO has everything done so far backed up what was originally planned

\chapter{Conclusion}
%TODO main takeaways so far

\chapter{Progress and the future}
%TODO what have i done so far (prototype, task backlog, design feedback), what needs to get done for the project to be finished

\bibliography{sources.bib}

\end{document}
