\section{Discussion}
\subsection{Limitation}

\subsubsection{Scope}

At every point in this projects design and development limitations of scope have been a constant cause of discussion and often dictated architectural or functional decisions. This is simply because the potential scope of the issue I'm trying to solve is extremely expansive, especially at the intersection of the real legal world and digital constant world. The main example is disputes which are essentially a disagreement over what is `real', because even though a \keyword{blockchain} or even just computers alone do keep a perfect record that record can always be \textit{wrong} but so can the real world. Is the computer mirroring the world or enforcing how the world should work?

\subsubsection{Implementation}
% TODO technical limitations, transfer and delete wallet ?

As a result of limited scope the final product is therefore limited in some of its implementation, this is functionality thats been built but hampered by scope and complexity. First is \textbf{gas}\footnote{gas is the cost in computation to process and complete a transaction on the \keyword{Ethereum} network which is taken from the senders account at the time of transaction, this is effectively a transaction fee like Visas roughly 1\% fee on debit transactions but based on work instead of transaction \textit{value}} which for simplicity is covered completely by the system meaning all transaction fees are paid by one account and users registering don't have to worry about paying to register, this is an attractive moral/ethical position but not business decision. Because this point is under discussion it can be seen as a limitation if the intent is for the system to charge for registration or a feature of a completely free and open system providing an essential utility coving all costs.

% TODO I have already talked about this is design and implementation do I need this?
Next is data consistency which is a problem faced by all distributed systems however severely more in a system that's distributed across domains the system doesn't control. This is exactly the situation this system is in, I can't control or own the \keyword{blockchain} just interact and ask it to do things. This is doubly compounded by the fact that everyone else has the exact level of power and control over the \keyword{blockchain} as I do, meaning anyone can interact with my \keyword{smart contract} directly without me. I discussed designing and implementing my \textit{ChainSync} solution for this in \ref{sec:chain-parity} and \ref{sec:chainSync}. % TODO Come back to this

\subsubsection{Social consensus/acceptance}

Currently \keyword{copyright} is backed by governments along with all other property law and more importantly has societies faith, trust is the largest barrier to success of this type of system. The question really is why would an author register their work with us over an established institution? This is a question of social change of which I'm no expert, therefore I can only express the benefits of this system (discussed mainly in section \ref{sec:intro}) as the reason for people to convert. Openness and independence I see as a strong benefit and I think a growing consensus of society agrees with me.  

\subsection{Blockchain technology}

\subsubsection{What are NFTs?}

NFTs are \keyword{smart contracts} that provide an interface for some immutable data (most commonly digital assets) on a \keyword{blockchain}, that interface includes ownership tracking, ownership transferring and access control management. The keyword here is \textit{non-fungible} which simply means not replaceable by another, an easy analogy for this is a \textit{bag of wheat} and a \textit{final project report}. The bag of wheat is fungible because one is easily replaced by another whereas you cant replace one final project report with another because they're unique to the project.

An NFT \keyword{smart contract} simply allows a user to save this non-fungible (unique) asset and assign themselves as the owner. This does not inherently mean the user who owns an NFT "owns" the asset unless specifically stated.

\subsubsection{Are these NFTs?}

Yes and no.
\br
Yes I took heavy inspiration from the \nft or NFT standard when developing my \keyword{smart contract} and yes the data they represent is definitely non-fungible (copyrightable works are by definition non-fungible).
\br
No my \keyword{smart contract} is not compatible with the \nft standard and therefore existing NFT products, NFTs are single owner while CRPL supports multiple owners. More importantly I believe the spirit of my contract is in contrast with NFTs which are being used overwhelmingly in \textit{get rich quick} schemes, fraud all with a heavy focus on money/value. The goal of my contract is to give protection of work to creators, at no point is money a direct focus other than protecting the value of a persons work. 

\subsubsection{How long will the blockchain last?}

The realist answer to this would be; probably not forever and almost definitely less time than a single \keyword{copyright} protection (100 years in this case). This is a pretty bleak outlook for my system, \keyword{blockchain} is still an infant technology even though by tech standards should be days past by now. Just like AI and machine learning I believe \keyword{blockchain} is at an incredible hight but also a make or break point, either the technology is completely accepted and integrated into the world or it's abandoned never to be seen again. After working with the technology for a couple months now I would bet on the latter outcome over engineering concerns but I've been wrong in the past.
