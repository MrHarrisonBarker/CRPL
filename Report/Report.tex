\documentclass[12pt]{article}

\usepackage[a4paper, margin=1in]{geometry}
\usepackage[english]{babel}
\usepackage[utf8]{inputenc}
\usepackage{hyperref}
\usepackage{subfigure}
%\usepackage{setspace}
%\usepackage{fancyhdr}


%\pagestyle{fancy}
%\fancyhf{}
%\fancyhead[L]{\rightmark}
%\fancyhead[R]{\thepage}
%\renewcommand{\headrulewidth}{0pt}
%\renewcommand{\sectionmark}[1]{\markboth{}{{\thesection~#1}}
%\renewcommand{\subsectionmark}[1]{}% Remove \subsection from header
%\onehalfspacing

\input{solidity-highlighting.tex}
\usepackage{listings, xcolor}

\definecolor{brilliantrose}{rgb}{1.0, 0.33, 0.64}
\definecolor{electriclime}{rgb}{0.8, 1.0, 0.0}
\definecolor{emerald}{rgb}{0.31, 0.78, 0.47}
\definecolor{amethyst}{rgb}{0.6, 0.4, 0.8}

\lstdefinelanguage{CSharp}
{
 morecomment = [l]{...},
 morecomment = [l]{//}, 
 morecomment = [l]{///},
 morecomment = [s]{/*}{*/},
 morestring=[b]", 
 commentstyle=\color{emerald}\ttfamily,
 sensitive = true,
 morekeywords = {abstract,  event,  new,  struct,
   as,  explicit,  null,  switch,
   base,  extern,  object,  this,
   bool,  false,  operator,  throw,
   break,  finally,  out,  true,
   byte,  fixed,  override,  try,
   case,  float,  params,  typeof,
   catch,  for,  private,  uint,
   char,  foreach,  protected,  ulong,
   checked,  goto,  public,  unchecked,
   class,  if,  readonly,  unsafe,
   const,  implicit,  ref,  ushort,
   continue,  in,  return,  using,
   decimal,  int,  sbyte,  virtual,
   default,  interface,  sealed,  volatile,
   delegate,  internal,  short,  void,
   do,  is,  sizeof,  while,
   double,  lock,  stackalloc, var, await,  
   else,  long,  static,   
   enum,  namespace,  string}
}


%Define the listing package
\usepackage{listings} %code highlighter
\usepackage{xcolor} %use color

%\lstset{
%    aboveskip=0cm,
%    stringstyle=\ttfamily,
%    showstringspaces = false,
%    basicstyle=\scriptsize\ttfamily,
%    commentstyle=\color{gray!45},
%    keywordstyle=\bfseries,
%    ndkeywordstyle=\bfseries,
%    identifierstyle=\ttfamily,
%    numbers=left,
%    numbersep=15pt,
%    numberstyle=\tiny,
%    numberfirstline = false,
%    breaklines=true
%}
%
%\lstdefinelanguage{JavaScript}{
%  keywords={typeof, new, true, false, catch, function, return, null, catch, switch, var, const, let, async, await, if, in, while, do, else, case, break, from},
%  ndkeywords={class, export, boolean, throw, implements, import, this},
%  sensitive=false,
%  comment=[l]{//},
%  morecomment=[s]{/*}{*/},
%  morestring=[b]',
%  morestring=[b]"
%}

\lstdefinelanguage{JavaScript}{
  morekeywords=[1]{break, continue, delete, else, for, function, if, in,
    new, return, this, typeof, var, void, while, with},
  % Literals, primitive types, and reference types.
  morekeywords=[2]{false, null, true, boolean, number, undefined,
    Array, Boolean, Date, Math, Number, String, Object},
  % Built-ins.
  morekeywords=[3]{eval, parseInt, parseFloat, escape, unescape},
  sensitive,
  morecomment=[s]{/*}{*/},
  morecomment=[l]//,
  morecomment=[s]{/**}{*/}, % JavaDoc style comments
  morestring=[b]',
  morestring=[b]"
}[keywords, comments, strings]

\usepackage{hyperref}
\hypersetup{
	colorlinks=true,
	allcolors=blue}

\title{CRPL: Interim Report}
\author{Harrison Beau Barker}

\usepackage{graphicx,float}
\graphicspath{ {images/} }

\usepackage{xcolor}

\usepackage{tabularx}
\newcolumntype{b}{X}
\newcolumntype{s}{>{\hsize=.5\hsize}X}

\newcommand{\keyword}[1]{\textbf{\textit{#1}}}
\newcommand{\q}[2]{\begin{quote}``#1''\cite{#2}\end{quote}}
\newcommand{\inlineQ}[2]{``#1''\cite{#2}}
\newcommand{\nft}[0]{\href{https://eips.ethereum.org/EIPS/eip-721}{EIP-721} }
\newcommand{\br}[0]{\\~\\}

\renewcommand{\thefootnote}{\roman{footnote}}

\setlength{\parskip}{1em}
%\setlength{\parindent}{1em}
\begin{document}

\begin{titlepage}
	\centering
	\includegraphics[width=0.4\textwidth]{crpl}\par
	\vspace{1cm}
	{\huge\bfseries CRPL: Report\par}
	\vspace{2cm}
	{\Large\itshape Harrison Beau Barker\par}
	{Supervised by Dr Dan McQuillan\par}
	\vfill
	{\url{https://github.com/MrHarrisonBarker/crpl}\par}
	\vspace{1cm}
	{\large Hbark002, 33575210\par}
\end{titlepage}

\abstract{CRPL (\textbf{C}opy\textbf{R}ight on a \textbf{P}ublic \textbf{L}edger) is a Ethereum blockchain backed system for registering and maintaining copyrights of digital works, this system is accompanied by a web app allowing users to easily interact with the system}

\tableofcontents{}

\section{Keywords}

\begin{description}
	\item[Copyright]
	\item[Blockchain]
	\item[Ethereum]
	\item[Smart Contract]
	\item[EVM] 
\end{description}

\section{Introduction}
\label{sec:intro}

The aim of this project is to represent a version of \keyword{copyright} for protection of intellectual property on a \keyword{blockchain} backed by a public ledger of transactions. My initial impetus for this project was a book I read in 2018 called \textit{"Blockchain Revolution: How the Technology Behind Bitcoin Is Changing Money, Business, and the World"}\cite{blockchain_revolution}, before this point I knew little of the applications for blockchain technologies attributing them only to a new form of digital currency allowing peer to peer transactions of wealth.

However, this book introduced the concept of \keyword{smart contracts} and the possibilities now small immutable programs could be saved and run on a blockchain. The most relevant change smart contracts brought was progressive state into an infamously immutable technology, by leveraging an unmodifiable chain of transactions state became not just a current record (like most traditional systems) but a historical account of all previous states with a clear a definable list of transformations precisely timestamped.

This new knowledge of \keyword{blockchains} came to fruition when it came to selecting my final project but to what problem should I help to provide a solution to leveraging this technology. A combination of recent interest in the crypto-sphere mainly coming from NFTs, continued displays of a \keyword{copyright} system not fit for purpose\cite{DMCA-abuse} and the book I had read 3 years earlier proposing that \keyword{blockchain} and \keyword{smart contracts} were an extremely viable solution to problems intersecting law and social structures.

\subsection{Unfit for purpose}
\label{sec:unfit}

So why is the current \keyword{copyright} system not fit for purpose? I've broken it into three interlinking factors; complexity, jurisdiction dependence and lack of digital computerised systems. starting with complexity, it is often hard as a creator to know if your work is protected or if the protection is enough? This gives massive power to publishers, managers and companies who are willing to exploit this fact by blinding a creative with a large contract covered in legal jargon. Should an artist also have to be a lawyer?

Jurisdiction dependence can be looked at as a point of complexity and inefficiency in the system, yes there's international \keyword{copyright} law in the form of the \textit{"Berne Convention"}\cite{Berne} and later \textit{"WIPO Copyright Treaty"}\cite{WIPO} which informs most of our modern \keyword{copyright} law. However, these treaties only state minimum requirements and standards to follow but do not control the internal \keyword{copyright} law of a sovereign nation which will always take precedence.

This idea of global \keyword{copyright} links nicely into my last factor which is centred around the digital and ever more interconnected world we're living in. Even though in 1996 the \textbf{WCT}\cite{WIPO} was introduced to address issues caused by the emergence of information technology, however the world and more specifically the internet has changed an unimaginable amount since 1996 but \keyword{copyright} is effectively unchanged including the systems to register and view registered \keyword{copyrights} which are closed off in obscurity.

\subsection{What a solution needs}

I've defined what I believe as to be four requirements of a solution to this problem and how I've addressed these requirements.

\begin{description}
	\item[Global] Both the system needed to be globally accessible, available and consistent across all jurisdictions to minimise complexity and maximise protection for an interconnected world. I've implemented this by using the \keyword{Ethereum} network which is made up of thousands of nodes across the world.
	\item[Open] Openness is essential to instil trust in a completely independent and alternative solution compared with government institutions that have implied trust and guaranty. I've implemented this by writing and using open-source code including \keyword{Ethereum} which is open-source and backed by a public ledger.
	\item[Robust] The current written laws and contracts maybe complex but are strongly defined with a commonly accepted interpretation, this will have to be true in this solution. I've implemented this by using an immutable \keyword{smart contract} for copyright representation and logic.
	\item[Simple] This is the most important requirement as the proposed problem is centred around current \keyword{copyright} complexity so any solution will need to be accessible to every possible user. I've implemented this by using clearly defined selectable \keyword{copyright} protections when registering a work.
\end{description}


\section{Research}
\subsection{Blockchain}

%\subsection{Blockchain}
%% TODO Technical knowledge needed to implement
%% TODO How it works?
%% TODO Why use it?
%
%\subsection{Existing solutions}
%
%\subsubsection{Copyright law}
%% TODO The state of the current copyright system
%\subsubsection{Online rights management}
%% TODO How do creators manage protection of their work
%
%\subsection{Development theory}
%% TODO Explanation of agile development methodology
%% TODO TDD and Scrum
%
%\subsection{Aims of the solution}
% TODO The over arcing aims of the solution, aka what should the system fix

The theory of a \keyword{blockchain} is not new \cite{origins_blockchain} but the implementation and hype around \keyword{blockchains} is new and extremely popular in the current day thanks to Bitcoin designed by Satoshi Nakamoto based on their white paper \cite{nakamoto2008bitcoin}. The accepted benefits of a \keyword{blockchain} are decentralisation, distribution, and immutability.For a \keyword{blockchain} to exist and be useful it needs nodes to maintain the data structure authenticate all transactions and hash the current block in preparation for the next in the chain.

\begin{figure}[H]
\caption{Blockchain structure with transactions saved in Merkle trees\cite{btc-white}}
\centering
\includegraphics[width=\textwidth,height=0.4\textheight,keepaspectratio]{images/patterns/merkle}
\centering
\end{figure}

\subsubsection{Technical proficiencies / limitations}

{\noindent\regular\textbf{Proficiencies}\vspace{2mm}}

\begin{description}
	\item[Security] Is the core of \keyword{blockchain} as the cryptographic hashing strategy is what makes the data structure a chain.
	\item[Transparency] Is not intrinsic to \keyword{blockchains} that is down to the implementation (public or private), however the underlying concept facilitates and promotes this open and transparent behaviour simply by the fact that nodes within a \keyword{blockchain} network need a copy of all transactions/blocks.
	\item[Decentralisation] Is definitely the most popular benefit of \keyword{blockchains} and it's not a technical benefit just like transparency decentralisation is a social debate that is made possible by technical decisions of the \keyword{blockchain} architecture, centred around the control and ownership of peoples data \begin{quote}"It avoids concentrations of power that could let a single person or organization take control."\cite{bohme2015bitcoin}\end{quote} 
\end{description}

{\noindent\regular\textbf{Limitations}\vspace{2mm}}

\noindent\keyword{Blockchain} technology faces two major limitations, the largest chains in the world (Bitcoin and \keyword{Ethereum}) are quickly growing in size which has pointed out scaleability problems caused by transaction speed and cost per transaction which are currently slow and expensive respectively compared to traditional systems (cost experiments have borne this out in the aptly named paper "Comparing blockchain and cloud services for business process execution"\cite{rimba2017comparing} and show that business logic on \keyword{blockchains} are 2 orders of magnitude more expensive than current cloud services).

\keyword{Blockchains} rely on \keyword{consensus protocols} which is a mathematical formula to determine consensus of the network, essentially enough of the nodes have to agree the current state of the \keyword{blockchain} whenever it's modified. Currently all major chains including Bitcoin and \keyword{Ethereum} use a protocol called proof of work\cite{PoW} which uses the total amount of compute a node has produced as the comparable proof that can be verified by many other nodes easily. This has worked so far for these major chains however ballooning energy consumption (which can be seen in realtime at the \href{https://ccaf.io/cbeci/index}{Cambridge Bitcoin Electricity Consumption Index}), high transaction costs, and inadequate transaction processing bandwidth is forcing them to change and look for solutions.

\keyword{Ethereum} have decided to completely reinvent their consensus protocol with a fundamentally different protocol called proof of stake which promises reduced energy consumption, cheaper, and faster transactions\cite{PoS}. Although there have been other proposed solutions including: improving the proof of stake system by parallelising it across nodes\cite{fi12080125} or reducing the block size for a increase transaction capacity by reducing the amount of work needed per block with a security trade off to balance\cite{kiayias2015speed}.

\subsubsection{Use of blockchain for IP protection}

I am not the first person to have thought of representing and enforcing \keyword{copyright} protection on a distributed \keyword{blockchain}, there has actually been a lot of discussion around the topic showing the benefits of using this technology for \keyword{copyright} such as transparency; \begin{quote} "Blockchain may substantially increase visibility and availability of information about copyright ownership." \cite{Copyright_in_the_blockchain_era} \end{quote} the power of \keyword{smart contracts} and utilising a networks cryptocurrency \begin{quote} "Smart contracts will allow automatic and instantaneous payments to designated parties, and expiration of a license after a certain amount of time." \cite{Copyright_in_the_blockchain_era} \end{quote} and simplification through the globalisation of \keyword{copyright} law. majority opinion says that introducing \keyword{smart contracts} to \keyword{blockchain} technology is extremely powerful particularly within \keyword{copyright} law to \begin{quote}"reliably automate a large volume of ‘dumb transactions’" \cite{missing_link_in_copyright_licensing}\end{quote} which will greatly reduce friction by removing unnecessary work and removing expertise needed currently in the field to be properly protected.

\subsubsection{Use of blockchain for legal purposes}

Bringing \keyword{copyright} to \keyword{blockchain} is apart of a larger conversation about the compatibility of any legal or even governmental workflows to be either represented or completely reinvented using \keyword{blockchain} technology, and it does look like in many cases these types of problems can leverage \keyword{blockchain} and possibly even thrive.

The legal industry is notoriously complicated requiring a great level of knowledge and expertise in the field to make sense of anything or more importantly get anything done. Logically laws make an obvious starting point for computerisation because computers are defined systems of ridged laws however this didn't make sense during the first wave of \keyword{blockchains} which almost entirely focused its efforts towards currency. The introduction of \keyword{smart contracts} has opened up the possibilities for automated law processing massively reducing boilerplate and bulky legal work which is largely trivial but time consuming.

\begin{quote}
	"So-called ‘smart contracts’ built on blockchain technologies may prove to be the most important example yet of “self-executing, customised rules”." \cite{MILLARD2018843}
\end{quote}

For governing \keyword{blockchains} ledger and transparency will take centre stage as essentially all governments are big collections of "things", assets, people and information. Not only are they collections but historical records, thankfully \keyword{blockchains} are immutable, timestamped, secure and built for openness. Because \keyword{blockchains} are just general purpose data structures there're uses are broad; \begin{quote}"keeping an overview of the authorities provided in a public organization and the ability to change the authority only if there is agreement among nodes which are classified as being higher ranked in the hierarchy." \cite{OLNES2017355}\end{quote}

% TODO Probably not needed, bit of a rehash of other sections

%\subsection{Copyright law}
%
%What is the reason for "fixing" \keyword{copyright}? is it broken? \keyword{Copyright} law is old, it was first introduced from the 1700s to the 1800s with the introduction of the printing press and the first true \keyword{copyright} act being the Statute of Anne in 1710 which has morphed into the Copyright, Designs and Patents Act 1988 in the UK (\href{https://www.legislation.gov.uk/ukpga/1988/48/contents}{see}) and the Copyright Act of 1976 in the US. The fact that each country has its own unique \keyword{copyright} laws (even though the international Berne Convention exists local jurisdictions still apply) exacerbated by the internet has been showing signs of cracks especially when "digital goods" are involved.
%
%Theses issues can be broken down into three key areas: 
%
%\begin{enumerate}
%	\item \textbf{Legal transparency/complexity} Is the absolutely at the top of this list, as a creative can you be sure your work is protected in every jurisdiction possible? No you can't because the current copyrighting system is a sparse disconnected collection of closed databases. Implementing a globally open database of copyrighted works with consistent rules and digital finger printing. 
%	\item \textbf{Royalty distributions (compensation)} Monetary transactions are the current forte of all major \keyword{blockchains} so why not simplify one of the hardest tasks for an creator, receiving payment for the use of a work of course this has become easier with the introduction of internet payments and purpose built systems like "PRS for Music" but integrating royalty payments straight into the \keyword{copyright} registration proves for secure and cohesive system.
%	\item \textbf{Cost} Of protection a work is essentially free as the Berne Convention prohibits the registration of a work as a requirement to protection (\keyword{copyright} is born with the work), however legally protecting and defending authorship of a work is not free and can be devastating not to mention extremely one sided (massive record company vs band). With the help of an immutable global ledger determining who registered the work is simple.
%\end{enumerate}
%
%These problems and solutions in this section are talked about in the paper \textit{Copyright in the blockchain era: Promises and challenges} \cite{Copyright_in_the_blockchain_era}

%\section{Environment and crypto} not enough for this section

\subsection{The market}

The vast majority of \keyword{blockchain} applications in the current day are financial for obvious reasons with the market capitalisation of cryptocurrencies being the universal metric for the \keyword{blockchain} markets size \cite{wood_2021}, however the health care industry has been aggressively investigating and implementing \keyword{blockchain} technology particularly in the secure distribution of medical data and health records \cite{8167115}, supply chain technology has seen some innovation tracking goods as they pass through a chain. 

\section{Design}

\subsection{Functional Requirements}
This is a list of key required features needed for this project to be successful.

\begin{figure}[H]
\caption{Functional requirements}
\begin{table}[H]
\begin{tabular}{|p{0.2\textwidth}|p{0.7\textwidth}|p{0.1\textwidth}|}
\hline
Feature                         & Function                                                                                                                                                        & Priority \\ \hline
Copyright smart contract        & Immutable code on a public ledger “blockchain“ for the purpose of establishing ownership or the copyright to a piece of work.                                   & CORE     \\ \hline
Multi-party distribution        & The ability to establish a complex ownership structure of multiple individuals/groups, all these owners will have equal ownership of the copyrighted work.      & CORE     \\ \hline
Ownership transfer              & The ability to change the ownership of a copyright from one complex structure to another with consent of all current owner(s).                                  & CORE     \\ \hline
Work verification               & Verification of a work to establish its originality with a reasonable accuracy for the platform, no duplicate works will be allowed to be registered.           & CORE     \\ \hline
Dispute filing                  & Allow any user to dispute a copyright with sufficient evidence and provide an option for resolving these disputes by the owner(s).                              & CORE     \\ \hline
Digital signing                 & Digitally sign a registered work with unique data that identifies it as original copyright registered, much like the traditional copyright registration symbol. & CORE     \\ \hline
Decentralised Work CDN \& proxy & Store registered works on a decentralised technology providing a free and open option for creators.                                                             & EXTRA    \\ \hline
Websocket updates               & Real-time updates for the front-end UI to provide a better end-user experience.                                                                                 & EXTRA    \\ \hline
\end{tabular}
\end{table}
\end{figure}

\subsection{Smart contract}

\subsubsection{Inspiration}

To design a smart contract without prior experience I decided look at what others had done before and because I knew my new contract was going to at least exhibit similar functionality and basic principles as \textbf{NFTs} I started with the EIP for non-fungible tokens \href{https://eips.ethereum.org/EIPS/eip-721}{EIP-721}. 

\begin{quote}
"Ethereum Improvement Proposals (EIPs) describe standards for the Ethereum platform, including core protocol specifications, client APIs, and contract standards."\cite{eip}
\end{quote}

This document describes a standard interface for all external methods and events an \textbf{NFT} contract should implement, most of these make sense straight away methods like: balanceOf, ownerOf, transferFrom and the Transfer event. Then there's a few methods to do with "approval" which is \keyword{Ethereum} language for access control, essentially what addresses are allowed to transact and make changes.

An interface spec is useful for understanding how the contract is supposed to interact with the outside world but nothing about how the contract works internally. So I went and found an implementation of this interface \href{https://github.com/OpenZeppelin/openzeppelin-contracts/blob/master/contracts/token/ERC721/ERC721.sol}{here} written by \href{https://github.com/OpenZeppelin}{OpenZeppelin} to gain an understanding of how these contracts operate.

\begin{figure}[H]
\caption{Structured Ownership essential mappings}
\centering
\begin{lstlisting}[language=Solidity]
	// Mapping from token ID to owner address
	mapping(uint256 => address) private _owners;

	// Mapping owner address to token count
	mapping(address => uint256) private _balances;

	// Mapping from token ID to approved address
	mapping(uint256 => address) private _tokenApprovals;

	// Mapping from owner to operator approvals
	mapping(address => mapping(address => bool)) private _operatorApprovals;
\end{lstlisting}
\end{figure}

This snippet is taken from the OpenZeppelin \href{https://github.com/OpenZeppelin/openzeppelin-contracts/blob/master/contracts/token/ERC721/ERC721.sol}{ERC721.sol} contract and is essentially how an \textbf{NFT} "works". a series of mappings that are saved in "storage" which is an area of the \keyword{EVM} that every smart contract has access to for storing state variables that need to persistent. These mappings are hashmaps from one type to another, first is the \_owners mapping which points to an owners address based on the hash of a given id (unit256). All the transactional methods are simply modifying these mappings, when you register a new token your wallet address is saved in the map entry for the next id.

\subsubsection{Ownership structure}

This core pattern/architecture was used as the foundation of my new contract, however there is one major requirement of my system not supported by the basic EIP-721 standard which is the ability of multiple address/people to have ownership of a token. This wasn't going to work for representing copyright as works can quite often involve multiple people collaborating the book I mentioned in the beginning of this report\cite{blockchain_revolution} has two authors, a system not representing the work and effort of all involved is just not acceptable.

\begin{figure}[H]
\caption{Structured Ownership essential mappings}
\centering
\includegraphics[width=0.5\textwidth,height=0.5\textheight,keepaspectratio]{images/operational/mappings.png}
\label{fig:float}
\end{figure}

To solve this problem I've redesigned how ownership is defined with the smart contract, instead of mapping the token id to one address the contract will now map to an \textbf{OwnershipStructure} which then points to a list of owner addresses along with a number of shares that specific address holds in the token.

This design obviously borrows a lot from limited companies share structure allowing for a complex ownership of multiple individuals or groups with implied variance in ownership (although the number of shares an address owns makes no immediate difference in the current implementation of this contract as this was outside of the desired complexity scope).

\subsubsection{Shareholder consensus}

Allowing multiple wallets to have ownership over a token now introduces a new problem for my contract design, when a change is made everyone has to agree to that change I can't just check if you're an owner anymore, giving the ability to change the copyright to everyone with a stake without consulting with all other owners is a point of exploitation.

\begin{figure}[H]
\caption{Structured Ownership proposal mappings}
\centering
\includegraphics[width=\textwidth,height=\textheight,keepaspectratio]{images/operational/prop-mappings.png}
\end{figure}

This is the solution I've designed for the shareholder consensus problem, now instead of making a direct change to the copyright (in this case an ownership restructure) a user proposes a change to the copyright which is then voted on by all the owners until a unanimous vote tally is reached then the change can be made.

\subsubsection{Protections}

To simplify and give users choice of the type of protection I designed a protections system similar to permission in many computer systems, the list of available protections was built from existing legal protections provided by \keyword{copyright} law \cite{rights_granted} including: Adaptation, Performance, Reproduction and Distribution. 

\subsection{Back-end}

\begin{figure}[H]
\caption{Back-end abstract operation}
\centering
\includegraphics[width=\textwidth,height=0.4\textheight,keepaspectratio]{images/operational/example-backend}
\centering
\end{figure}

\subsubsection{Dependancy injection}

% bit iffy could be better?
Dependancy injection is a supported and heavily encouraged design pattern within \textbf{.NET} and \textbf{ASP.NET} which allows for building loosely coupled applications by separating out implementation and design and the ability to depend on a softwares design opposed to its technical implementation allows for more resilient and modular code less dependant on a specific implementation.

\begin{figure}[H]
\caption{DI graph taken from \href{https://docs.microsoft.com/en-us/dotnet/architecture/modern-web-apps-azure/architectural-principles#dependency-inversion}{docs.microsoft}}
\centering
\includegraphics[width=\textwidth,height=\textheight,keepaspectratio]{images/patterns/ms-di}
\centering
\end{figure}

This graph shows a generic example of inversion of control and dependancy injection, as you can see each class is depending on an interface of the desired class not the actual code implementation hence loosely coupled.

\begin{figure}[H]
\caption{Example dependancy graph for the query controller and service}
\centering
\includegraphics[width=0.7\textwidth,height=0.7\textheight,keepaspectratio]{images/patterns/DI-example}
\end{figure}

This is real example of dependancy injection drawn from the query controller and service injections. It shows that the two classes don't import any other implemented classes just the interfaces describing how you can interact with an implemented version of that class. At run time each interface injected will be populated from the service collection with an implemented version of the class thats be registered on startup.

\subsubsection{Background services}

The design of background services was built on previous work I wrote for \href{https://github.com/mrharrisonbarker/openevent}{OpenEvent} so instead of trying new software like \href{https://www.hangfire.io/}{Hangfire} which does look more feature rich and well used, however I decided for the scale of this project and the already existing pattern I had developed and knew intimately a year ago would be a better fit.

\begin{figure}[H]
\caption{Queued background service pattern}
\centering
\includegraphics[width=0.5\textwidth,height=0.5\textheight,keepaspectratio]{images/patterns/background-processing-pattern}
\end{figure}

This is the basic design pattern describing how my background services work, it's essentially made up of a queue and processing service. "Work" is queued while a processing service running in its own thread dequeues work then processes accordingly. Once the work has been finished the processing service waits for the next item on the queue.

This pattern can be quickly implemented and tailored to a specific type of background work, it's easily scalable with the number of threads available and settable in the processing service.

\subsubsection{Event processing}

Because transactions on the blockchain can technically take any amount of time to be verified, placed into a block and then for that block to be placed onto the chain. Blocks on \keyword{Ethereum} are processed around every 13 seconds and you're not guaranteed to be placed into the next block which largely depends on the amount of \textbf{gas} you're willing to spend and the number of transactions currently being processed by the network.

All this means that my system has to be able to send a transaction then wait an indeterminate amount of time for a response and I can't force the user to wait on that transaction until complete. Thankfully \keyword{Ethereum} has a solution for this problem called \textbf{Events}, I've specified a number of these events in my contract definition (see below) which the system then "listens" for by processing the information in each block.

\begin{figure}[H]
\caption{\href{https://github.com/MrHarrisonBarker/CRPL/blob/main/CRPL.Contracts/contracts/IStructuredOwnership.sol}{IStructuredOwnership} events}
\begin{lstlisting}[language=Solidity]
/// @dev Emits when a new copyright is registered
event Registered(uint256 indexed rightId, OwnershipStake[] to);

/// @dev Emits when a copyright has been restructured and bound
event Restructured(uint256 indexed rightId, RestructureProposal proposal);

/// @dev Emits when a restructure is proposed
event ProposedRestructure(uint256 indexed rightId, RestructureProposal proposal);

/// @dev Emits when a restructure vote fails
event FailedProposal(uint256 indexed rightId);
\end{lstlisting}
\end{figure}

When an event is found it gets added to a processing queue then a processing service dequeues each event and processes based on the type of event (this uses the background service pattern discussed in the previous section).

\begin{figure}[H]
\caption{Blockchain event listeners and processing}
\centering
\includegraphics[width=\textwidth,height=0.5\textheight,keepaspectratio]{images/operational/Event-Listening}
\end{figure}

\subsubsection{Applications framework}
% TODO Data model structure (application, view model, input model)

Handling forms and applications is awkward and full of edge cases, the code for handing these applications (eg: copyright registration) can become large and convoluted especially when your system implements many. For this system five applications are needed: copyright registration, ownership restructure, dispute, wallet transfer and delete account. I decided to design a solution for handling application flow and state as a generic process, this means all applications will follow the same state flow and interaction endpoints \textit{seen below.} 

\begin{figure}[H]
\caption{Applications framework state diagram}
\centering
\includegraphics[width=\textwidth,height=0.5\textheight,keepaspectratio]{images/operational/applications-status}
\end{figure}

This generification and ridged design flow proved extremely useful, this was because keeping a clear view of state is essential when keeping parity between my system and the outer \keyword{blockchain}.

\subsection{Database}

The database for this system needs to store two types of data: data stored on both the \keyword{blockchain} and CRPL (eg: public wallet addresses, contract address, and registered works) and data stored solely on the database not mirrored with the chain (eg: applications, user account information and explicit database relationships).

\begin{figure}[H]
\caption{Database EER diagram}
\centering
\includegraphics[width=\textwidth,height=0.7\textheight,keepaspectratio]{images/patterns/database}
\end{figure}

\subsubsection{Chain parity}
\label{sec:chain-parity}

Data need to be kept in parity with the \keyword{blockchain} introduces a problem that being data on the chain can change independently of the system, a user could transact with the \keyword{copyright} \keyword{smart contract} and register a new copyright or propose a new ownership structure and even bind that new structure completely destroying any continuity between my database and what's real. This will lead to a terrible user experience but there's nothing I can do about it, the chain is open to anyone and is quite literally the whole point of this project.

However I can react to change, again everything is open and accessible all I have to do is read the \keyword{blockchain} and update my database accordingly keeping in mind that the chain is always the one source of truth not my database.

\subsubsection{Independent from the chain}

I've chosen to keep certain data off the \keyword{blockchain} only representing it on the CRPL database. Of course technically I could store everything on chain completely independent of any database, however this would ballon the size of my \keyword{smart contract} which are limited to 24KB complied it would also create a problem for maintainability and future development. 
Imagine everything is represented in the smart contract including dispute handling and applications, the contract is verified onto the chain and is now running in the \keyword{EVM} what happens when I want to add a new type of application or I find out my dispute handling is un-ethical or exploitable? I can't change the smart contract its immutable if I really wanted to I could deploy a new contract and manually migrate all previous \keyword{copyrights} to this new contract, however this would be expensive, slow and arguably against the spirit of \keyword{blockchain} technology and the law as I'm effectively changing the underlying representation of a users \keyword{copyright} without their consultation or approval.

\subsection{Front-end}

Visual design of the web application was on the lowest priority a focus on pure usability and function was always the priority because of the complex undertaking that was needed to implement all functionality building effectively two backend systems (\keyword{blockchain} and web API).

I decided to use the \href{https://clarity.design/}{Clarity} design system and libraries as they have support for Angular and has an enterprise/function first focus.

\begin{figure}[H]
\caption{Original dashboard page wireframe}
\centering
\includegraphics[width=\textwidth,height=0.5\textheight,keepaspectratio]{images/wireframe/Dashboard}
\end{figure}

\begin{figure}[H]
\caption{Final dashboard design}
\centering
\includegraphics[width=\textwidth,height=0.5\textheight,keepaspectratio]{images/wireframe/dashboard-real}
\end{figure}

\begin{figure}[H]
\caption{Original register form wireframe}
\centering
\includegraphics[width=\textwidth,height=0.5\textheight,keepaspectratio]{images/wireframe/Register}
\end{figure}

\begin{figure}[H]
\caption{Final register design}
\centering
\includegraphics[width=\textwidth,height=0.5\textheight,keepaspectratio]{images/wireframe/register-real}
\end{figure}

\subsection{Architecture}
% TODO an overview of the system architecture design

\subsection{Development process}

Development was split into four sprints, two larger sprints two weeks long to start development focusing on major core features essential for any success of the project: \keyword{smart contracts}, contract interaction, registration and restructure applications and user authentication. Followed up with two one week sprints focusing on secondary features and quality of life: disputes, search, synchronisation and account config. 

\begin{figure}[H]
\hfil
\begin{tabular}{|l|l|l|l|}
\hline
Sprint & Start            & End              & Length  \\ \hline
1      & 19 January 2022  & 2 February 2022  & 2 weeks \\ \hline
2      & 4 February 2022  & 18 February 2022 & 2 weeks \\ \hline
3      & 20 February 2022 & 27 February 2022 & 1 week  \\ \hline
4      & 1 March 2022     & 8 March 2022     & 1 week  \\ \hline
\end{tabular}
\end{figure}

\section{Implementation}

\subsection{Smart contract}

\subsubsection{Interface overview}

The contract interface is split in two, \href{https://github.com/MrHarrisonBarker/CRPL/blob/main/CRPL.Contracts/contracts/IStructuredOwnership.sol}{IStructuredOwnership.sol} which handles multi-ownership transactions and events and \href{https://github.com/MrHarrisonBarker/CRPL/blob/main/CRPL.Contracts/contracts/ICopyright.sol}{ICopyright.sol} which handles all \keyword{copyright} transactions and events (similar to the \href{https://eips.ethereum.org/EIPS/eip-721}{EIP-721} interface).

\begin{figure}[H]
\caption{\href{https://github.com/MrHarrisonBarker/CRPL/blob/main/CRPL.Contracts/contracts/ICopyright.sol}{ICopyright} events}
\begin{lstlisting}[language=Solidity]
/// @dev Emits when a new address is approved to a copyright
event Approved(uint256 indexed rightId, address indexed approved);

/// @dev Emits when a new manager has been approved
event ApprovedManager(address indexed owner, address indexed manager, bool hasApproval);
\end{lstlisting}
\end{figure}

\begin{figure}[H]
\caption{\href{https://github.com/MrHarrisonBarker/CRPL/blob/main/CRPL.Contracts/contracts/IStructuredOwnership.sol}{IStructuredOwnership} events}
\centering
\begin{lstlisting}[language=Solidity]
/// @dev Emits when a new copyright is registered
event Registered(uint256 indexed rightId, OwnershipStake[] to);

/// @dev Emits when a copyright has been restructured and bound
event Restructured(uint256 indexed rightId, RestructureProposal proposal);

/// @dev Emits when a restructure is proposed
event ProposedRestructure(uint256 indexed rightId, RestructureProposal proposal);

/// @dev Emits when a restructure vote fails
event FailedProposal(uint256 indexed rightId);
\end{lstlisting}
\end{figure}

% TODO come back if any words left to describe each event?
First looking at all the events most are straight forward, \textit{Approved} and \textit{ApproveManager} are taken from the \nft standard (ApproveManager == ApprovalForAll) then for \textit{IStructuredOwnership} I've introduced four new events enabling modifiable multi address ownership with consensus.

\begin{figure}[H]
\caption{\href{https://github.com/MrHarrisonBarker/CRPL/blob/main/CRPL.Contracts/contracts/IStructuredOwnership.sol}{IStructuredOwnership} functions}
\centering
\begin{lstlisting}[language=Solidity]
/// @notice The current ownership structure of a copyright
/// @param rightId The copyright id
function OwnershipOf(uint256 rightId) external view returns (OwnershipStake[] memory);

/// @notice Proposes a restructure of the ownership share of a copyright contract, this change must be bound by all share holders
/// @param rightId The copyright id
/// @param restructured The new ownership shares
/// @param notes Any notes written concerning restructure for public record
function ProposeRestructure(uint256 rightId, OwnershipStake[] memory restructured) external payable;

/// @notice The current restructure proposal for a copyright
/// @param rightId The copyright id
/// @return A restructure proposal
function Proposal(uint256 rightId) external view returns (RestructureProposal memory);
    
function CurrentVotes(uint256 rightId) external view returns (ProposalVote[] memory);

/// @notice Binds a shareholders vote to a restructure
/// @param rightId The copyright id
/// @param accepted If the shareholder accepts the restructure
function BindRestructure(uint256 rightId, bool accepted) external payable;
\end{lstlisting}
\end{figure}

Four new and one modified functions enable this functionality, \textit{ProposeRestructure} to propose a change of ownership and \textit{BindRestructure} to vote for proposals. \textit{OwnershipOf} is modified \textit{ownerOf} from \nft which originally returns an address but now returns a complex ownership structure.

\subsubsection{Registration}

Registration of a \keyword{copyright} as discussed in the \ref{sec:smart-contract-design} uses basic principles of existing contract implementations most importantly the \nft contract, which registers new tokens by assigning a wallet address in a token ids mapping entry stored on the contract. This is simple and secure as the resulting hash of a token id will always point to the same entry therefore address.

\begin{figure}[H]
\caption{Essential copyright mappings from \href{https://github.com/MrHarrisonBarker/CRPL/blob/main/CRPL.Contracts/contracts/Copyrights/CopyrightBase.sol}{CopyrightBase.sol}}
\centering
\begin{lstlisting}[language=Solidity]
// rightId -> ownership structures
mapping (uint256 => OwnershipStructure) internal _shareholders;
    
// rightId -> metadata
mapping(uint256 => Meta) internal _metadata;
\end{lstlisting}
\end{figure}

I've taken this design and modified it for my needs by mapping to a complex ownership structure instead of one address (enabling multi-ownership), I've then added a new mapping to save metadata (work hash, expiry date, legal protections) for each \keyword{copyright}. These technically could be merged into one mapping along with the four others, mapping from id to a larger more complex struct encompassing all data needed. However I wanted to keep the size of my data structures small with their own defined purpose, this also reduces transaction costs because only the data relevant is being processed.

\begin{figure}[H]
\caption{Register function from \href{https://github.com/MrHarrisonBarker/CRPL/blob/main/CRPL.Contracts/contracts/Copyrights/CopyrightBase.sol}{CopyrightBase.sol}}
\centering
\begin{lstlisting}[language=Solidity]
function Register(OwnershipStake[] memory to, Meta memory meta) public validShareholders(to) {
	
	uint256 rightId = _copyCount.next();

	// registering copyright across all shareholders
	for (uint256 i = 0; i < to.length; i++) {

		require(to[i].share > 0, INVALID_SHARE);

		_recordRight(to[i].owner);
		_shareholders[rightId].stakes.push(to[i]);
	}
        
	_metadata[rightId] = meta;
	_shareholders[rightId].exists = true;
        
	_approvedAddress[_copyCount.getCurrent()] = msg.sender;

	emit Registered(rightId, to);
	emit Approved(rightId, msg.sender);
}	
\end{lstlisting}
\end{figure}

As you can see from the above registration is straightforward: generate the next id, iterate over all shareholders inputted, add each shareholder to the mapping checking each has some shares, add metadata to mapping, approve the sender for this \keyword{copyright} and emit a registered and approved event to let the system know whats happened.

Lastly I want to focus on line 8, \textit{require} is apart of \textbf{Solidity}'s error handling. The \keyword{EVM} runs all functions transactionally in the programming sense meaning all changes to the persistent data structure are processed and saved after the function has completed successfully. This is extremely useful and greatly simplifies error handling because if we encounter an error no underlying changes to the data have taken place and the transaction simply fails. 

\textit{Require} therefore checks an expression is true and if not an error is thrown with a stated reason the transaction is canceled and no changes to stored data structures are made. These are used throughout my \keyword{smart contract} to validate input data, I talk more about these in \ref{sec:modifiers} below. This \textit{require} on line 8 is checking all shareholders have more than zero shares otherwise throw an error with "INVALID\_SHARE".

The events emitted from this function are "listened" to and processed in the back-end more information in section \ref{sec:blockchain-event-listeners}.

\subsubsection{Ownership restructure}

As discussed in \keyword{smart contract} design I had to build a shareholder consensus function from scratch for making changes to the \keyword{copyright}, therefore changing the ownership structure is split into two functions/steps, first you propose a new structure then each shareholder must vote or \textit{"bind"} the new structure, when all the shareholders have agreed to the new structure mappings are updated to reflect the new ownership.

\begin{figure}[H]
\caption{ProposeRestructure function from \href{https://github.com/MrHarrisonBarker/CRPL/blob/main/CRPL.Contracts/contracts/Copyrights/CopyrightBase.sol}{CopyrightBase.sol}}
\centering
\begin{lstlisting}[language=Solidity]
function ProposeRestructure(uint256 rightId, OwnershipStake[] memory restructured) external override validId(rightId) isExpired(rightId) validShareholders(restructured) isShareholderOrApproved(rightId, msg.sender) payable {
        
        for (uint256 i = 0; i < restructured.length; i++) {

            require(restructured[i].share > 0, INVALID_SHARE);

            _newHolders[rightId].stakes.push(restructured[i]);
            _newHolders[rightId].exists = true;
        }   

        emit ProposedRestructure(rightId, _getProposedRestructure(rightId));
    }	
\end{lstlisting}
\end{figure}

Above is the first step which is the proposal, this is a small function that iterates through all the new shareholders and adds their address and shares to a new mapping called \textit{\_newHolders} which is the same as the \textit{\_shareholders} mapping and is there to hold proposed ownership structures before they're "bound". An event is emitted telling the back-end the proposal has been saved to the chain and to start accepting votes.

\begin{figure}[H]
\caption{BindRestructure function from \href{https://github.com/MrHarrisonBarker/CRPL/blob/main/CRPL.Contracts/contracts/Copyrights/CopyrightBase.sol}{CopyrightBase.sol}}
\centering
\begin{lstlisting}[language=Solidity]
function BindRestructure(uint256 rightId, bool accepted) external override validId(rightId) isExpired(rightId) isShareholderOrApproved(rightId, msg.sender) payable 
{
	_checkHasVoted(rightId, msg.sender);
     
	// record vote
	_proposalVotes[rightId].push(ProposalVote(msg.sender, accepted));
	_numOfPropVotes[rightId] ++;

	for (uint256 i = 0; i < _proposalVotes[rightId].length; i ++) 
	{
		if (!_proposalVotes[rightId][i].accepted) 
		{
			_resetProposal(rightId);
			emit FailedProposal(rightId);

			return;
		}
	}

	// if the proposal has enough votes, **** 100% SHAREHOLDER CONSENSUS ****
	if (_numOfPropVotes[rightId] == _numberOfShareholder(rightId)) {
            
		// proposal has been accepted and is now binding

		OwnershipStake[] memory oldOwnership = OwnershipOf(rightId);
            
		// reset has to happen before new shareholders are registered to remove data concerning old shareholders
		_resetProposal(rightId);

		_shareholders[rightId] = _newHolders[rightId];

		delete(_newHolders[rightId]);

		emit Restructured(rightId, RestructureProposal({oldStructure: oldOwnership, newStructure: OwnershipOf(rightId)}));
	}
}	
\end{lstlisting}
\end{figure}

Now looking at the more complex \textit{BindRestructure} function. First the address is checked if they've voted already, the vote is then recorded in a new mapping \textit{\_proposalVotes}, all votes are checked, if any of the votes are false then the whole proposal is rejected, checks if all the shareholders have voted, if all the votes are in set \textit{\_shareholders} to the proposed structure from \textit{\_newHolders} and emit an event.
\br
This is a point of possible future development or discussion, for a proposal to be \textit{"bound"} a unanimous vote is needed however this system could take advantage of the distribution of shares with only a majority of shares needed to make a change.

I decided to keep the voting unanimous over concerns of complexity (extension of development time was predicted) and a possible moral issues as to the possibilities of exploitation using this. An issue of exploitation is inherent to both implementations however a unanimous vote gives equal power of exploitation to every party whereas using shares would give more power to high staked parties.

\subsubsection{Modifiers}
\label{sec:modifiers}

Modifiers are pieces of code run at either the end or start of a function call usually used to verify function parameters, I'm using them exclusively at the start of function calls to test addresses, ids and expiry. The design of a modifier usually consists of an assertion plus any processing needed on the data. 

\begin{figure}[H]
\caption{isShareholderOrApproved modifier from \href{https://github.com/MrHarrisonBarker/CRPL/blob/main/CRPL.Contracts/contracts/Copyrights/CopyrightBase.sol}{CopyrightBase.sol}}
\centering
\begin{lstlisting}[language=Solidity]
modifier isShareholderOrApproved(uint256 rightId, address addr) 
{
	uint256 c = 0;
	for (uint256 i = 0; i < _shareholders[rightId].stakes.length; i++) 
	{
		if (_shareholders[rightId].stakes[i].owner == addr) c ++;
	}
	require(c == 1 || _approvedAddress[rightId] == addr, NOT_SHAREHOLDER);
	_;
}
\end{lstlisting}
\end{figure}

This modifier authenticates a message sender is allowed to make changes to a specific \keyword{copyright} by checking it exists in ether \textit{\_shareholders} or \textit{\_approvedAddress} mappings.

\begin{figure}[H]
\caption{validAddress modifier from \href{https://github.com/MrHarrisonBarker/CRPL/blob/main/CRPL.Contracts/contracts/Copyrights/CopyrightBase.sol}{CopyrightBase.sol}}
\centering
\begin{lstlisting}[language=Solidity]
modifier validAddress(address addr)
{
	require(addr != address(0), INVALID_ADDR);
	_;
}
\end{lstlisting}
\end{figure}

\keyword{Ethereum} has a reserved address called the \textit{zero-address} only used for \keyword{smart contract} creation transactions, it's good practice to check all the addresses the contract handles are not the \textit{zero-address}. 

\begin{figure}[H]
\caption{isExpired modifier from \href{https://github.com/MrHarrisonBarker/CRPL/blob/main/CRPL.Contracts/contracts/Copyrights/CopyrightBase.sol}{CopyrightBase.sol}}
\centering
\begin{lstlisting}[language=Solidity]
modifier isExpired(uint256 rightId)
{
	require(_metadata[rightId].expires > block.timestamp, EXPIRED);
	_;
}
\end{lstlisting}
\end{figure}

\keyword{Copyright} expiry is handled with a modifier instead of an explicit state change as it's not feasible to have a timed process that runs when the expiry time is hit because of the time scale \keyword{copyright} works in. Therefore every time a transaction interacts with a \keyword{copyright} the expiry time is checked against the current block timestamp throwing an error code if expired.

\subsection{Back-end}

The back-end comprises of an \textbf{ASP.NET} API and static file server, the API is comprised of controllers that depend on services which hold business logic.

\begin{figure}[H]
\caption{HTTP request generic data flow}
\centering
\includegraphics[width=\textwidth,height=\textheight,keepaspectratio]{images/patterns/service-controller}
\end{figure}

\subsubsection{API overview}
\begin{figure}[H]
\caption{Forms controller endpoints}
\centering
\includegraphics[width=\textwidth,height=0.5\textheight,keepaspectratio]{images/operational/Forms-Api}
\end{figure}

looking at the most complex and largest controller I tried to keep it as \textit{RESTful} as possible with all endpoints describing the resource accessed in conjunction with appropriate HTTP methods describing the action performed.

\subsubsection{Applications framework}

The applications framework exploits object oriented programming techniques mainly inheritance and polymorphism. Application have three classes each an implementation of each abstract base class, this means applications can be handled interchangeably generically and individually based on the child class.

\begin{figure}[H]
\caption{Application base class \href{https://github.com/MrHarrisonBarker/CRPL/blob/main/CRPL.Data/Applications/DataModels/Application.cs}{derived from}}
\centering
\includegraphics[width=\textwidth,height=0.5\textheight,keepaspectratio]{images/operational/application-base}
\end{figure}

The base \textit{Application} class is used by \textbf{EF Core} to generate database migrations creating and modifying tables, therefore this class establishes all relationships in this case a many-to-many with users and one-to-many with a registered work.

\begin{figure}[H]
\caption{Application view model class \href{https://github.com/MrHarrisonBarker/CRPL/blob/main/CRPL.Data/Applications/ViewModels/ApplicationViewModel.cs}{derived from}}
\centering
\includegraphics[width=\textwidth,height=0.5\textheight,keepaspectratio]{images/operational/application-view}
\end{figure}

Application view models are used when retrieving and displaying data, a lot of the time you don't want to send everything from the base model to the client or some processing/mapping is needed.
 
\begin{figure}[H]
\caption{Application input model class \href{https://github.com/MrHarrisonBarker/CRPL/blob/main/CRPL.Data/Applications/InputModels/ApplicationInputModel.cs}{derived from}}
\centering
\includegraphics[width=\textwidth,height=0.5\textheight,keepaspectratio]{images/operational/application-input}
\end{figure}

Lastly input models that are used when updating applications, this usually represents a real form the user is interacting with. This model is interpreted when updating which modifies the data model on the database.
\br
One huge benefit of all applications using the same base class is that state can be generalised and kept consistent between different types of applications, a completed registration application should logically be the same as a completed dispute application (This state was discussed in design section \ref{sec:application-framework-design}).

\begin{figure}[H]
\caption{Example application updater \href{https://github.com/MrHarrisonBarker/CRPL/blob/main/CRPL.Web/Core/Applications/Updaters/CopyrightRegistrationUpdater.cs}{CopyrightRegistrationUpdater}}
\centering
\begin{lstlisting}[language=CSharp]
public static class CopyrightRegistrationUpdater
{
    private static readonly List<string> Encodables = new() { "OwnershipStakes", "Id"  };
    
    public static async Task<CopyrightRegistrationApplication> Update(this CopyrightRegistrationApplication application, CopyrightRegistrationInputModel inputModel, IServiceProvider serviceProvider)
    {
        var userService = serviceProvider.GetRequiredService<IUserService>();
        
        application.UpdateProperties(inputModel, Encodables);

        if (inputModel.OwnershipStakes != null)
        {
            application.OwnershipStakes = inputModel.OwnershipStakes.Encode();

            application.CheckAndAssignStakes(userService, inputModel.OwnershipStakes);
        }

        return application;
    }
}
\end{lstlisting} 
\end{figure}

Each application has an updater which is used for parsing the input model and updating the data model, this one is for the copyright registration application and is fairly simple. First it gets an instance of the user service, calls a static method I made that matches class properties between the input model and data model then updates those properties using the input model, complex object can't be saved in a db column so the ownership stakes are encoded into strings and a relationship is updated/established between all the shareholders and the application.

\begin{figure}[H]
\caption{Example application submitter \href{https://github.com/MrHarrisonBarker/CRPL/blob/main/CRPL.Web/Core/Applications/Submitters/CopyrightRegistrationSubmitter.cs}{CopyrightRegistrationSubmitter}}
\centering
\begin{lstlisting}[language=CSharp]
public static class CopyrightRegistrationSubmitter
{
    public static async Task<CopyrightRegistrationApplication> Submit(this CopyrightRegistrationApplication copyrightRegistrationApplication, IServiceProvider serviceProvider)
    {
        var registrationService = serviceProvider.GetRequiredService<IRegistrationService>();
        
        await registrationService.StartRegistration(copyrightRegistrationApplication);

        copyrightRegistrationApplication.Status = ApplicationStatus.Submitted;
        
        return copyrightRegistrationApplication;
    }
}
\end{lstlisting}
\end{figure}

An application also needs a submitter for when all the data has been inputted, in the case of \textit{CopyrightRegistrationSubmitter} it starts the registration process (creates work and starts verifying) and sets the status of the application to submitted. The majority of submitters start some background process or will require further processing to then be completed.

\subsubsection{Registration process}

The copyright registration process can be broken down into: fill and submit application, verification, send to chain and transaction verified.

\begin{figure}[H]
\caption{Registration state flow}
\centering
\includegraphics[width=\textwidth,height=0.5\textheight,keepaspectratio]{images/operational/cpy-registration-status-graph}
\end{figure}

The first step works as previously discussed, a user fills out a form/application submitting when all data has been inputted. validation of the data is handled on the client side with some server side sanity checks, this was to reduce development workload and complexity however it's compromised by a user calling the API directly either maliciously or if the API was made public.

% There is a section below talking about this step
\begin{figure}[H]
\caption{Finding collisions \href{https://github.com/MrHarrisonBarker/CRPL/blob/main/CRPL.Web/Services/WorksVerificationService.cs}{Line 57:58}}
\centering
\begin{lstlisting}[language=CSharp]
var collision = await Context.RegisteredWorks
           	.FirstOrDefaultAsync(x => x.Hash == work.Hash && x.Id != workId);
\end{lstlisting}
\end{figure}

Verification is handled by a background service working though a queue which checks for collisions by comparing the uploaded work hash against all known work hashes. If a collision is found the application is set to \textit{Failed} and the work is set to \textit{Rejected}.

\begin{figure}[H]
\caption{Register message \href{https://github.com/MrHarrisonBarker/CRPL/blob/main/CRPL.Web/Services/RegistrationService.cs}{Line 88:102}}
\centering
\begin{lstlisting}[language=CSharp]
var register = new RegisterFunction()
{
	To = application.OwnershipStakes.Decode().Select(x => Mapper.Map<OwnershipStakeContract>(x)).ToList(),
	Meta = new Meta
	{
		Expires = new BigInteger((DateTime.Now.AddYears(application.YearsExpire) - new DateTime(1970, 1, 1, 0, 0, 0, DateTimeKind.Utc)).TotalSeconds),
		Title = application.Title,
		Registered = new BigInteger((DateTime.Now - new DateTime(1970, 1, 1, 0, 0, 0, DateTimeKind.Utc)).TotalSeconds),
		LegalMeta = application.Legal,
		WorkHash = Encoding.UTF8.GetString(application.WorkHash),
		WorkUri = application.WorkUri,
		WorkType = application.WorkType.ToString(),
		Protections = application.Protections
	}
};
\end{lstlisting}
\end{figure}

Once verified a \keyword{copyright} owner then has to `complete' the registration which sends a register message (seen above) to the chain returning the transaction hash for later reference.

\begin{figure}[H]
\caption{Registered Event Processor \href{https://github.com/MrHarrisonBarker/CRPL/blob/main/CRPL.Web/Core/EventProcessors/RegisteredEventProcessor.cs}{Line 29:31}}
\centering
\begin{lstlisting}[language=CSharp]
context.Update(registeredWork);

registeredWork.RightId = registeredEvent.Event.RightId.ToString();
registeredWork.Registered = DateTime.Now;
registeredWork.Status = RegisteredWorkStatus.Registered;

registeredWork.AssociatedApplication.First(x => x.ApplicationType == ApplicationType.CopyrightRegistration).Status = ApplicationStatus.Complete;	
\end{lstlisting}
\end{figure}

Nodes on the network will now process this transactions taking an indeterminate time based on how much money you're willing to spend and the current capacity of the network. Once verified onto the chain a \textit{Registered} event will be picked up by the event processors running, the processor will match the transaction hashes setting the work to registered and its registration application to complete.

\subsubsection{Queries - Chain injection}

\begin{figure}[H]
\caption{Data injection from \keyword{blockchain} when querying registered works}
\centering
\includegraphics[width=\textwidth,height=\textheight,keepaspectratio]{images/operational/chain-inject}
\end{figure}

Because I rely on the \keyword{blockchain} as the store and point of truth in the system I require a way of querying a \keyword{copyright} for data to display. 

\begin{figure}[H]
\caption{Metadata query from query service \href{https://github.com/MrHarrisonBarker/CRPL/blob/main/CRPL.Web/Services/QueryService.cs}{Line 155:156}}
\centering
\begin{lstlisting}[language=CSharp]
var meta = await new Contracts.Copyright.CopyrightService(BlockchainConnection.Web3(), ContractRepository.DeployedContract(CopyrightContract.Copyright).Address)
		.CopyrightMetaQueryAsync(rightId);
...
registeredWork.Meta = meta != null ? meta.ReturnValue1 : null;
\end{lstlisting}
\end{figure}

This is a snippet of a function called \textit{injectFromChain} found in the \href{https://github.com/MrHarrisonBarker/CRPL/blob/main/CRPL.Web/Services/QueryService.cs}{query service} and is called when work is requested from the service, it queries the \keyword{blockchain} for data then "injects" the result into the existing registered work data model.

\subsubsection{Dispute handling}
% TODO resolving disputes: payment, change of ownership

\begin{figure}[H]
\caption{Dispute file and resolve flow diagram}
\centering
\includegraphics[width=\textwidth,height=\textheight,keepaspectratio]{images/operational/dispute-workflow}
\end{figure}

Disputes are applications with an \textit{ExpectedRecourse} and will not complete until the attached recourse is handled. Disputes are publicly viewable once submitted to maintain transparency and reduce exploitation. 

\subsubsection{Blockchain event listeners}
\label{sec:blockchain-event-listeners}

To listen for events in the \keyword{blockchain} \textbf{Nethereum} has a class called \textit{BlockchainProcessor} which listens for an event and registers a callback that is runs when that specific event is found.

\begin{figure}[H]
\caption{Registering an event listener for \textit{Registered} event \href{https://github.com/MrHarrisonBarker/CRPL/blob/main/CRPL.Web/Services/Background/BlockchainEventListener.cs}{Line 32:33}}
\centering
\begin{lstlisting}[language=CSharp]
BlockchainConnection.Web3().Processing.Logs
	.CreateProcessorForContract<RegisteredEventDTO>(ContractRepository.DeployedContract(CopyrightContract.Copyright).Address, log => EventQueue.QueueEvent(log))
\end{lstlisting}
\end{figure}

I'm using event processors that listen to a specific event type from a specific \keyword{smart contract}, see above this processor is listening for events of type \textit{RegisteredEventDTO} from the deployed contract retrieved from the \textit{ContractRepository}.

All my processors push all events to the event queue to be processed by a service instead of processing in the callback, this increases scaleability and breaks the code down into smaller modular chunks.

Events are pulled from the queue by the \href{https://github.com/MrHarrisonBarker/CRPL/blob/main/CRPL.Web/Services/Background/EventProcessingService.cs}{EventProcessingService} and processed by an \href{https://github.com/MrHarrisonBarker/CRPL/tree/main/CRPL.Web/Core/EventProcessors}{EventProcessor} which are static classes with one method \textit{ProcessEvent}, one is built for every event type listened to. 

\begin{figure}[H]
\caption{Processing events based on type \href{https://github.com/MrHarrisonBarker/CRPL/blob/main/CRPL.Web/Services/Background/EventProcessingService.cs}{Line 32:49}}
\centering
\begin{lstlisting}[language=CSharp]
switch (nextEvent.GetType().FullName)
{
	case var name when name.Contains("RegisteredEvent"):
		await ((EventLog<RegisteredEventDTO>)nextEvent).ProcessEvent(ServiceProvider, Logger);
		break;
	case var name when name.Contains("ApprovedEvent"):
		await ((EventLog<ApprovedEventDTO>)nextEvent).ProcessEvent(ServiceProvider, Logger);
		break;
	case var name when name.Contains("ProposedRestructureEvent"):
		await ((EventLog<ProposedRestructureEventDTO>)nextEvent).ProcessEvent(ServiceProvider, Logger);
		break;
	case var name when name.Contains("RestructuredEvent"):
		await ((EventLog<RestructuredEventDTO>)nextEvent).ProcessEvent(ServiceProvider, Logger);
		break;
	case var name when name.Contains("FailedProposalEvent"):
		await ((EventLog<FailedProposalEventDTO>)nextEvent).ProcessEvent(ServiceProvider, Logger);
		break;
}
\end{lstlisting}
\end{figure}

\subsubsection{Verification pipeline}

The verification pipeline uses my background service architecture with a \textit{VerificationQueue} and \textit{VerificationPipelineService} that dequeues and processes verification of a work. 

\begin{figure}[H]
\caption{Verification pipeline}
\centering
\includegraphics[width=\textwidth,height=\textheight,keepaspectratio]{images/operational/verification-pipe}
\end{figure}

When a work is dequeued it is verified using the \href{https://github.com/MrHarrisonBarker/CRPL/blob/main/CRPL.Web/Services/WorksVerificationService.cs}{WorksVerificationService} method \textit{VerifyWork} that searches the database for existing works with the same hash. If no collisions are found the works status is updated to \textit{Verified}, if collision are found then the work is set to \textit{Rejected}

\begin{figure}[H]
\caption{hashing a work \href{https://github.com/MrHarrisonBarker/CRPL/blob/main/CRPL.Web/Services/WorksVerificationService.cs}{Line 146:151}}
\centering
\begin{lstlisting}[language=CSharp]
private byte[] HashWork(byte[] work)
{
	using var hashAlgorithm = SHA512.Create();

	return hashAlgorithm.ComputeHash(work);
}
\end{lstlisting}
\end{figure}

For hashing the uploaded work I stream the file into a byte array and compute the hash using \textbf{SHA-512} which has \(2^{512}\) total possible hash outputs which is \(1.34 \times 10^{154}\) almost double the number of atoms in the universe at around \(10^{82}\) so the possibility of a collision or exploitation is extremely low. 

Although \textbf{MD5} is mostly commonly used for very quick file duplicity checks these checks are usually just checksums to check file integrity of downloaded files. In terms of cryptographic security \textbf{SHA-512} is far better, \textbf{MD5} has been `broken' for years so should never be used for any sensitive or secure data hashing where as the latest \textbf{SHA} algorithms are considered secure and used in conjunction with "salt" for passwords in many systems.

\subsubsection{Digital signing}

Digital signing is handled at the point of registration by \href{https://github.com/MrHarrisonBarker/CRPL/tree/main/CRPL.Web/Core/WorkSigners}{WorkSigners} built for specific file types. Additionally there's a universal signer applied to all works and simply concatenates a series of unique bytes to the end of the file stream.

\begin{figure}[H]
\caption{\href{https://github.com/MrHarrisonBarker/CRPL/blob/main/CRPL.Web/Core/WorkSigners/UniversalSigner.cs}{UniversalSigner.cs}}
\centering
\begin{lstlisting}[language=CSharp]
public CachedWork Sign(CachedWork work)
{
	byte[] signature = Encoding.UTF8.GetBytes(Signature);

	return new CachedWork
	{
		Work = work.Work.Concat(signature).ToArray(),
		ContentType = work.ContentType
	};
}
\end{lstlisting}
\end{figure}

\subsection{Database}

I've chosen to use a \textbf{MySQL} database for this project for two reasons: ease of use and \textbf{ACID}. Having previously used \textbf{MySQL} for multiple projects in the past plus the brilliant integration between \textbf{EFCore}, \textbf{LINQ} and \textbf{SQL}. \textbf{ACID} standing for \textbf{a}tomicity, \textbf{c}onsistency, \textbf{i}solation and \textbf{d}urability which ensures data integrity at the cost of speed over a \textit{NoSql} approach like \textbf{MongoDB}, it also has support for rigid relationships between entities.

\subsubsection{ChainSync}
\label{sec:chainSync}

As talked about previously some data has to be kept in parity with the \keyword{Blockchain} by querying the deployed contract.

\begin{figure}[H]
\caption{ChainSync}
\centering
\includegraphics[width=\textwidth,height=0.5\textheight,keepaspectratio]{images/operational/chain-sync}
\end{figure}

Therefore a background service was built that synchronises all the \keyword{copyrights} currently being tracked with the \keyword{blockchain}, the service checks for changes on startup and then every 24 hours after startup.

\begin{figure}[H]
\caption{24 hour sync timer \href{https://github.com/MrHarrisonBarker/CRPL/blob/main/CRPL.Web/Core/ChainSync/ChainSyncService.cs}{Line 25:30}}
\centering
\begin{lstlisting}[language=CSharp]
CronTimer = new Timer(
	Sync,
	null,
	TimeSpan.Zero,
	TimeSpan.FromHours(24)
);
\end{lstlisting}
\end{figure}

There's one synchroniser implemented and it's for ownership structures, it queries the \keyword{blockchain} using \textit{OwnershipOf} which is a method on the contract that returns an ownership structure for a specific \keyword{copyright}. It then checks against what is saved in the database, if changes are found relationships need to be removed and updated.

\begin{figure}[H]
\caption{ownership similarity \href{https://github.com/MrHarrisonBarker/CRPL/blob/main/CRPL.Web/Core/ChainSync/Synchronisers/OwnershipSynchroniser.cs}{Line 71, 77:86}}
\centering
\begin{lstlisting}[language=CSharp]
if (ownerships.Count != work.UserWorks.Count || !work.UserWorks.All(x => ownerships.Contains(x.UserAccount.Wallet.PublicAddress.ToLower())))
...
work.UserWorks.Clear();
ownerships.ForEach(async owner =>
{
	var user = await Context.UserAccounts.FirstOrDefaultAsync(x => x.Wallet.PublicAddress.ToLower() == owner.ToLower());
	if (user == null) throw new UserNotFoundException(owner);
	work.UserWorks.Add(new UserWork()
	{
		UserAccount = user
	});
});
\end{lstlisting}
\end{figure}

\subsection{Front-end}

A fancy UI was not a high priority as the core back-end and \keyword{blockchain} interaction is expansive and complex enough in isolation, however I wanted to at least give a user the ability to interact with the system I've built hopefully in a usable and accessible way.

Most user interface design thought and work was spent on the four key forms (user register, copyright register, ownership restructure and dispute) as from previous experience forms are the hardest piece (for me) of interface design by far. To make this easier I counter intuitively at first did not care about the design or usability of the components I was building only getting the data into the form and sent off to the back-end.

This produced some initial designs that functionally work but were confusing users and just didn't look very nice, I've found it's very easy to get hung up on user interface when building systems\footnote{I think this is down to the fact that it's the single point of contact with your users, most people will not see and or care about business logic code but how your interface looks and feels are front and centre. this would be okay in a real product development setting but for this project it's just not within the scope.} which slows down development especially in the early stages of development when the constraints and needs of the system are changing.

Then after all core features had been built I went back to my forms and re-designed all input components using a consistent set of rules, requirements and style.    

\begin{figure}[H]
\caption{Example input markup \href{https://github.com/MrHarrisonBarker/CRPL/blob/main/CRPL.Web/ClientApp/src/app/Forms/cpy-registration-form/cpy-registration-form.component.html}{cpy-registration-form.component.html 9:18}}
\centering
\begin{lstlisting}[language=html]
<div class="input-container">
	<label class="input-label">Title<sup>*</sup></label>
	<div class="input-control">
	<input type="text" name="title" formControlName="Title" placeholder="Hello world"/>
	<div>This is a searchable title for the copyright, it doesn't have to be unique only descriptive and relevant&nbsp;<sup>*saved to the blockchain</sup></div>
		<clr-alert class="input-error" [clrAlertClosable]="false" clrAlertType="danger" *ngIf="InvalidAndUntouched('Title')">
		 	<clr-alert-item>This field is required!</clr-alert-item>
	 </clr-alert>
	</div>
</div>
\end{lstlisting}
\end{figure}

Below is the my \href{https://github.com/MrHarrisonBarker/CRPL/tree/main/CRPL.Web/ClientApp/src/app/Forms/ownership-structure-form}{ownership-structure-form} component that's used in both \href{https://github.com/MrHarrisonBarker/CRPL/tree/main/CRPL.Web/ClientApp/src/app/Forms/cpy-registration-form}{cpy-registration-form} and \href{https://github.com/MrHarrisonBarker/CRPL/tree/main/CRPL.Web/ClientApp/src/app/Forms/cpy-restructure-form}{cpy-restructure-form}, this means all forms have to be consistent across the application so any section of a form will `fit' into any other form.

\begin{figure}[H]
\caption{\href{https://github.com/MrHarrisonBarker/CRPL/tree/main/CRPL.Web/ClientApp/src/app/Forms/ownership-structure-form}{Ownership structure form} component}
\centering
\includegraphics[width=\textwidth,height=\textheight,keepaspectratio]{images/wireframe/ownership-structure}
\end{figure}

\section{Discussion}
\subsection{Limitation}

\subsubsection{Scope}

At every point in this projects design and development limitations of scope have been a constant cause of discussion and often dictated architectural or functional decisions. This is simply because the potential scope of the issue I'm trying to solve is extremely expansive, especially at the intersection of the real legal world and digital constant world. The main example is disputes which are essentially a disagreement over what is `real', because even though a \keyword{blockchain} or even just computers alone do keep a perfect record that record can always be \textit{wrong} but so can the real world. Is the computer mirroring the world or enforcing how the world should work?

\subsubsection{Implementation}
% TODO technical limitations, transfer and delete wallet ?

As a result of limited scope the final product is therefore limited in some of its implementation, this is functionality thats been built but hampered by scope and complexity. First is \textbf{gas}\footnote{gas is the cost in computation to process and complete a transaction on the \keyword{Ethereum} network which is taken from the senders account at the time of transaction, this is effectively a transaction fee like \textbf{Visas} roughly 1\% fee on debit transactions but based on work instead of transaction value.} which for simplicity is covered completely by the system meaning all transaction fees are paid by one account and users registering don't have to worry about paying to register, this is an attractive moral/ethical position but not business decision. Because this point is under discussion it can be seen as a limitation if the intent is for the system to charge for registration or a feature of a completely free and open system providing an essential utility coving all costs.

The live version of this \href{https://crpl.azurewebsites.net/}{project} is deployed to the \href{https://ethereum.org/en/developers/docs/networks/}{Ropsten Testnet} at \href{https://ropsten.etherscan.io/tx/0x427c6335df3af15aa9d6ba0da22a2e25025badb3f8f5c38d8399d10d1e7db3c6}{0x427c6335...} costing a total of \textit{0.004893532690548546 Ether} and using \textit{2,409,858 gas} which works out to £11.28 (based on a £2,304.45 per Ether). Although this is not representative of costs on \textbf{Mainnet}, to calculate this cost I can however use the gas used\footnote{\textbf{Ropsten} uses the same proof-of-work system as \textbf{Mainnet} so gas calculations will be very similar.} in conjunction with the current gas price.

\begin{figure}[H]
\caption{Contract deployment cost at different gas prices based on \href{https://etherscan.io/gastracker}{gastracker}}
\begin{table}[H]
\centering
\begin{tabular}{p{0.2\textwidth}p{0.2\textwidth}p{0.2\textwidth}p{0.2\textwidth}}
\hline
\multicolumn{2}{|l|}{\multirow{2}{*}{\begin{tabular}[c]{@{}l@{}}Gas used\\  2,409,858\end{tabular}}} & \multicolumn{2}{l|}{\multirow{2}{*}{\begin{tabular}[c]{@{}l@{}}Price per Ether\\  £2,304.45\end{tabular}}} \\
\multicolumn{2}{|l|}{} & \multicolumn{2}{l|}{} \\ \hline
 &  &  &  \\ \cline{2-4} 
\multicolumn{1}{l|}{} & \multicolumn{1}{l|}{Gas price} & \multicolumn{1}{l|}{Fee (Eth)} & \multicolumn{1}{l|}{Fee (£)} \\ \hline
\multicolumn{1}{|l|}{Low} & \multicolumn{1}{l|}{12} & \multicolumn{1}{l|}{0.028918296} & \multicolumn{1}{l|}{£66.64} \\ \hline
\multicolumn{1}{|l|}{Average} & \multicolumn{1}{l|}{50} & \multicolumn{1}{l|}{0.1204929} & \multicolumn{1}{l|}{£277.67} \\ \hline
\multicolumn{1}{|l|}{High} & \multicolumn{1}{l|}{200} & \multicolumn{1}{l|}{0.4819716} & \multicolumn{1}{l|}{£1,110.68} \\ \hline                                    
\end{tabular}
\end{table}
\end{figure}

Next is data consistency which is a problem faced by all distributed systems however severely more in a system that's distributed across domains the system doesn't control. This is exactly the situation this system is in, I can't control or own the \keyword{blockchain} just interact and ask it to do things. This is doubly compounded by the fact that everyone else has the exact level of power and control over the \keyword{blockchain} as I do, meaning anyone can interact with my \keyword{smart contract} directly without me. I discussed designing and implementing my \textit{ChainSync} solution for this in \ref{sec:chain-parity} and \ref{sec:chainSync}.

\subsubsection{Social consensus/acceptance}

Currently \keyword{copyright} is backed by governments along with all other property law and more importantly has societies faith, trust is the largest barrier to success of this type of system. The question really is why would an author register their work with us over an established institution? This is a question of social change of which I'm no expert, therefore I can only express the benefits of this system (discussed mainly in section \ref{sec:intro}) as the reason for people to convert. Openness and independence I see as a strong benefit and I think a growing consensus of society agrees with me.  

\subsection{Blockchain technology}

\subsubsection{What are NFTs?}

NFTs are \keyword{smart contracts} that provide an interface for some immutable data (most commonly digital assets) on a \keyword{blockchain}, that interface includes ownership tracking, ownership transferring and access control management. The keyword here is \textit{non-fungible} which simply means not replaceable by another, an easy analogy for this is a \textit{bag of wheat} and a \textit{final project report}. The bag of wheat is fungible because one is easily replaced by another whereas you cant replace one final project report with another because they're unique to the project.

An NFT \keyword{smart contract} simply allows a user to save this non-fungible (unique) asset and assign themselves as the owner. This does not inherently mean the user who owns an NFT "owns" the asset unless specifically stated.

\subsubsection{Are these NFTs?}

Yes and no.
\br
Yes I took heavy inspiration from the \nft or NFT standard when developing my \keyword{smart contract} and yes the data they represent is definitely non-fungible (copyrightable works are by definition non-fungible).
\br
No my \keyword{smart contract} is not compatible with the \nft standard and therefore existing NFT products, NFTs are single owner while CRPL supports multiple owners. More importantly I believe the spirit of my contract is in contrast with NFTs which are being used overwhelmingly in \textit{get rich quick} schemes, fraud all with a heavy focus on money/value. The goal of my contract is to give protection of work to creators, at no point is money a direct focus other than protecting the value of a persons work. 

\subsubsection{How long will the blockchain last?}

The realist answer to this would be; probably not forever and almost definitely less time than a single \keyword{copyright} protection (100 years in this case). This is a pretty bleak outlook for my system, \keyword{blockchain} is still an infant technology even though by tech standards should be days past by now. Just like AI and machine learning I believe \keyword{blockchain} is at an incredible height but also a make or break point, either the technology is completely accepted and integrated into the world or it's abandoned never to be seen again. After working with the technology for a couple months now I would bet on the latter outcome over engineering concerns but I've been wrong in the past.


\section{Evaluation}
\subsection{Process}
Evaluating the process of designing and building a system.

\subsubsection{Design/Pre-Development} 

Software design and the ratio of time spent on design over development is a debated topic within the industry, the most popular theory currently being \textit{Agile} which is biased towards development over design whereas \textit{Waterfall} considered a legacy methodology believes in designing the entire solution before development can take place.

I focused my design on gaining knowledge of the technologies I had no previous experience: \keyword{blockchain} interaction and \keyword{smart contract} design. Only abstract models of the system were drawn so I could research all the needed technology.

% maybe some pictures of my early diagrams, in notebook.

In practice this strategy was reasonably successful as the overall architecture is representative of my original design, in spite of this many key systems were underestimated or poorly designed largely thanks to lack of knowledge. Although this is how agile development is supposed to work, it's okay if fundamental change happens in fact you should always consider change and change quickly.

\subsubsection{Development}

As far as timing development was perfect, every deadline was hit on time with no need for extension or reevaluation. I put this down to aggressive and continuous scope handling, it's very easy as an engineer to completely over-engineer a system and I've had problems managing this desire in the past. However for this project I clearly defined the scope in my mind so whenever new complexity was being debated I always forced myself to consider scope and time constraints.

As a result if you read all my \href{https://github.com/MrHarrisonBarker/CRPL/blob/main/README.md}{sprint reviews} and burndown graphs I consistently hit deadlines and manage the amount of work assigned to each sprint. 

In conjunction with agile development I also initially planned to use a development workflow called \textbf{TDD} (\textbf{T}est \textbf{D}riven \textbf{D}evelopment) that says functionality should first be test cases the developer then aims to pass by implementing functionality. It's very popular in the \textit{Agile} space and can increase code quality and speed up development, however functionally limited when used in larger user focused and very infrastructure heavy systems. I still tested heavily but my implementation wasn't dependent on testing.

\begin{figure}[H]
\caption{example burndown for sprint 2}
\centering
\includegraphics[width=\textwidth,height=\textheight,keepaspectratio]{images/burndown-2}
\end{figure}

\subsubsection{Was it agile?}
% TODO is this needed

To assess if my process has been agile Ill being using the 4 values from the agile manifesto\cite{agile}

\begin{description}
	\item[Individuals and interactions over processes and tools] isn't very applicable as this is an individual project.
	\item[Working software over comprehensive documentation] links back to my troubles with \textbf{TDD} I was focusing more on passing the test than what the software need to do. 
	\item[Customer collaboration over contract negotiation] is an area I could've done a lot more in, my customer is an imagined aggregate of certain struggles not a real person this could've been a better representation with more hands on research.
	\item[Responding to change over following a plan] would be a great summary of this project because although I had a fairly comprehensive plan I only every took it as a guide, If I'd spent weeks on a master plan it would've only been a waste of time.
\end{description}

\subsection{Product}
Evaluating the system that has been built functionally and non-functionally.

\subsubsection{Functional specification}

Have I built a system that satisfies my specification?

\begin{figure}[H]
\caption{success of each functional requirement}
\begin{table}[H]
\begin{tabular}{|p{0.2\textwidth}|p{0.8\textwidth}|}
\hline
Feature                         & Successfully implemented?                                                                                                                                                                                                                                                                                                                                                           \\ \hline
Copyright smart contract        & A fully featured copyright contract has been built and deployed onto a test blockchain, the contract is capable of registering a copyrighted work with sufficient metadata to establish ownership and specific legal protections.                                                                                                                                                   \\ \hline
Multi-party distribution        & Ownership of a registered work is represented as a multi-party share structure allowing many authors to have "ownership" of a single work.                                                                                                                                                                                                                                          \\ \hline
Ownership transfer              & The ownership of a registered work can be changed via a proposal and vote system requiring the unanimous consensus of all current owners. *Voting is not majority share-based like limited companies.                                                                                                                                                                               \\ \hline
Work verification               & Verification of works is currently simple using hash comparison, this finds complete file accurate copies and so therefore even simply adding a zero or null data to the file bypasses verification. To solve this problem more complex algorithms will be needed or the use of third-party services, this was outside the scope of this project.                                   \\ \hline
Dispute filing                  & The system allows anyone (except the owner) to dispute a registered work for a given reason and the choice of an expected recourse (ownership change or payment). Right owners can then either reject the dispute or accept and apply the expected recourse.                                                                                                                        \\ \hline
Digital signing                 & Once a work has been registered we inject metadata into the uploaded file for proof of registration, this is done two ways: custom singers built for supported file types and a universal signer that inserts data at the end of any file type.                                                                                                                                     \\ \hline
Decentralised Work CDN \& proxy & All registered works are automatically saved to IPFS (InterPlanetary File System) which is a peer to peer network for storing files distributed securely in chunks over multiple nodes. A user can then access this file either through an official gateway (ipfs.io/ipfs/), running your own node to connect to the network or my public gateway (ipfs.harrisonbarker.co.uk/ipfs). \\ \hline
Websocket updates               & All updates are done through a single WebSocket, users are subscribed to works and applications when they click on them or have ownership associated with them.  The implementation of this is crude though as it was an extra feature added minimally towards the end of the development cycle with many concerns of scaling issues.                                               \\ \hline

\end{tabular}
\end{table}
\end{figure}

\subsubsection{Is it fit for purpose}

Have I built a system that addresses my issues with \keyword{copyright} discussed in \autoref{sec:unfit}? My stated three factors were \textit{complexity, jurisdiction dependence and lack of digital computerised systems}, does my system solve any of these?

First is complexity, by removing as much legal jargon from the process and a customisable protections system users know exactly how their work is protected without the need for legal council or manager therefore empowering creators. 

Next is jurisdiction dependence which is an escapable fact of the legal and governmental world that can cause issue and confusion when faced with a global internet culture. My system is global both because of \keyword{Ethereum} and its independence from any government.

% TODO doesn't get the point across enough
Lastly digitalisation/computerisation, my entire system is essentially computerisation of \keyword{copyright} law which helps to solve my last two points but more importantly it's built for copyrightable digital work which traditional \keyword{copyright} is lacking in. Standardised digital signing and verification is proof of this ambition.

% TODO Is this implementation ethical? (more ethical than traditional)

\subsubsection{Testing}

% How was the project tested (smart contracts, backend, frontend middleware, no ui frontend tests)
The system has been extensively unit tested at three levels: \keyword{smart contract}, back-end and front-end middleware.

Because contracts run within the \keyword{EVM} a testing environment is needed in this case \href{https://hardhat.org/}{Hardhat} which allows me to deploy and run tests on a \keyword{smart contract}.

\begin{figure}[H]
\caption{\href{https://github.com/MrHarrisonBarker/CRPL/blob/main/CRPL.Contracts/test/Copyright/Register.ts}{Register.ts} unit test snippet}
\centering
\begin{lstlisting}[language=JavaScript]
it('Should REVERT no shareholders', async function ()
{
	await expect(deployedContract.Register([], meta)).to.be.revertedWith('NO_SHAREHOLDERS');
});

it('Should REVERT when invalid shareholders', async function ()
{
	await expect(deployedContract.Register([{owner: ethers.constants.AddressZero, share: 1}], meta)).to.be.revertedWith('INVALID_ADDR');
});
\end{lstlisting}
\end{figure}

This is what the tests look like there're then run outputting statistics for each payable transaction on the contract including deployment.

\begin{figure}[H]
\caption{contract statistics from Hardhat}
\centering
\includegraphics[width=\textwidth,height=0.4\textheight,keepaspectratio]{images/appendix/tests/hardhat-results}
\centering
\end{figure}

Back-end tests are written in the \textit{CRPL.Tests} project separated by domain all using \href{https://nunit.org/}{NUnit} test framework and runner.

\begin{figure}[H]
\caption{Back-end test directory structure}
\hfill
\subfigure{\includegraphics[width=\textwidth,height=0.5\textheight,keepaspectratio]{images/appendix/tests/test-tree-1}}
\hfill
\subfigure{\includegraphics[width=\textwidth,height=0.3\textheight,keepaspectratio]{images/appendix/tests/test-tree-2}}
\hfill
\end{figure}

The front-end logic is tested using Angular's built in test framework \href{https://jasmine.github.io/}{Jasmine} found in \textit{.spec.ts} files.

\begin{figure}[H]
\caption{\href{https://github.com/MrHarrisonBarker/CRPL/blob/main/CRPL.Web/ClientApp/src/app/_Services/auth.service.spec.ts}{auth.service.spec.ts} nonce fetch test}
\centering
\begin{lstlisting}[language=JavaScript]
it('should fetch nonce', inject(
      [HttpTestingController, AuthService],
      (httpMock: HttpTestingController, authService: AuthService) =>
      {
        let mockNonce: string = "TEST NONCE";
        service['Address'] = 'TEST ADDRESS';

        authService.fetchNonce().subscribe((nonce: string) =>
        {
          expect(nonce).toEqual("TEST NONCE")
        });

        let request = httpMock.expectOne("user/nonce?walletAddress=TEST%20ADDRESS");

        expect(request.request.responseType).toEqual('json');
        expect(request.cancelled).toBeFalsy();

        request.flush(mockNonce);
      }
    )
  );
\end{lstlisting}
\end{figure}

\textbf{All test reuslts can be found in appendix \ref{sec:test-results}}

% What is the result of the tests




\section{Future development}

Some potential future development avenues

\begin{description}
	\item[Royalty payments] Royalty payment in cryptocurrency based on the number of shares a user owns in the copyright, automatic based on usage and fairly distributed.
	\item[Marketplace] An open market for purchasing the legal right to use a work from the rights holder directly.
	\item[External/Advanced verification] Currently work verification/authentication is very basic and can be exploited, proving authorship is a hard but essential issue to solve.
	\item[Analytics] provide users with data describing the use of their work.
\end{description}

\section{Conclusion}

The aim of this project was to successfully represent basic \keyword{copyright} for works on a blockchain while also providing a web application allowing users to interact and maintain \keyword{copyrights}. This system proves the possibility of such a system through its functionality, however it's not a complete system ready for market as the legal and social complexities of the problem are expansive that need to be taken seriously and a level of scrutiny not possible in the defined scope.

So to conclude while the technical operation of this project fulfils the stated specification it doesn't prove that a blockchain representation of \keyword{copyright} is the \textit{correct} solution but stand as compelling evidence towards assessing \keyword{blockchain} as a viable solution.

%\section{Appendix}

\subsection{Acknowledgements}

\begin{table}[H]
\begin{tabular}{|p{0.2\textwidth}|p{0.6\textwidth}|p{0.05\textwidth}|p{0.15\textwidth}|}
\hline
Name           & Use                                                                                   &                                               & Subsystem \\ \hline
Rider          & Primary IDE for development                                                           & \href{https://www.jetbrains.com/rider}{link}                  & DEV       \\ \hline
YouTrack       & Issue tracking software                                                               & \href{https://www.jetbrains.com/youtrack}{link}               & DEV       \\ \hline
VSCode         & Editor used for developing smart contracts                                            & \href{https://code.visualstudio.com}{link}                     & DEV       \\ \hline
RemixIDE       & Web based IDE for developing, testing, deploying and interacting with smart contracts & \href{https://remix.ethereum.org/}{link}                       & DEV       \\ \hline
Azure DevOps   & CD/CI pipeline for running tests and deploying to Azure                               & \href{https://azure.microsoft.com/en-us/services/devops}{link} & DEV       \\ \hline
Azure          & Cloud based deployment                                                                & \href{https://azure.microsoft.com/}{link}                      & DEV       \\ \hline
Ethereum       & Public open source blockchain software with smart contract support                    & \href{https://ethereum.org}{link}                              & BLOCK     \\ \hline
Solidity       & Smart contract programming compiler                                                   & \href{https://docs.soliditylang.org}{link}                     & BLOCK     \\ \hline
Metamask       & Browser extension based Ethereum wallet with over 30 million users                    & \href{https://metamask.io}{link}                               & BLOCK     \\ \hline
Hardhat        & Smart contract development environment                                                & \href{https://hardhat.org}{link}                               & BLOCK     \\ \hline
Waffle         & Smart contract testing framework                                                      & \href{https://getwaffle.io}{link}                              & BLOCK     \\ \hline
ASP.NET        & Web application/service framework for .NET                                            & \href{https://dotnet.microsoft.com/en-us/apps/aspnet}{link}    & BACK-END  \\ \hline
.NET           & Cross platform development framework for C\#                                          & \href{https://dotnet.microsoft.com/en-us/}{link}               & BACK-END  \\ \hline
EFCore         & Object database mapper for .NET                                                       & \href{https://docs.microsoft.com/en-us/ef/core/}{link}         & BACK-END  \\ \hline
Nethereum      & Ethereum interaction library for .NET                                                 & \href{https://nethereum.com/}{link}                            & BACK-END  \\ \hline
NUnit          & .NET test runner                                                                      & \href{https://nunit.org/}{link}                                & BACK-END  \\ \hline
AutoMapper     & .NET object mapping library                                                           & \href{https://automapper.org/}{link}                           & BACK-END  \\ \hline
Cronos         & Cron job handling library                                                             & \href{https://www.nuget.org/packages/Cronos/}{link}            & BACK-END  \\ \hline
iTextSharp     & PDF interaction and modification library used for digital signing                     & \href{https://www.nuget.org/packages/iTextSharp/}{link}        & BACK-END  \\ \hline
TagLibSharp    & Exif tag library used for digital signing                                             & \href{https://github.com/mono/taglib-sharp}{link}              & BACK-END  \\ \hline
ImageSharp     & Image metadata library used for digital signing                                       & \href{https://github.com/SixLabors/ImageSharp}{link}           & BACK-END  \\ \hline
TypeScript     & JavaScript superset providing strong typing                                           & \href{https://www.typescriptlang.org/}{link}                   & FRONT-END \\ \hline
Angular        & JavaScript framework for development of single page dynamic web applications          & \href{https://angular.io/}{link}                               & FRONT-END \\ \hline
Clarity Design & CSS and JavaScript design framework                                                   & \href{https://clarity.design/}{link}                           & FRONT-END \\ \hline
RxJs           & Library for asynchronous JavaScript programming                                       & \href{https://rxjs.dev/}{link}                                 & FRONT-END \\ \hline
Signalr        & Web-socket platform                                                                   & \href{https://www.npmjs.com/package/@microsoft/signalr}{link}  & FRONT-END \\ \hline
web3.js        & Ethereum JavaScript API library                                                       & \href{https://www.npmjs.com/package/web3}{link}                & FRONT-END \\ \hline
\end{tabular}
\end{table}

\subsection{Design docs}
\subsection{Sprint reviews}
\subsection{User guide}



\subsection{Test results}
\label{sec:test-results}



\bibliographystyle{plain}
\bibliography{sources.bib}

\end{document}
