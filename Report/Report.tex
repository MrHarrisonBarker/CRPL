\documentclass[12pt]{article}

\usepackage[a4paper, margin=1in]{geometry}
\usepackage[english]{babel}
\usepackage[utf8]{inputenc}

\usepackage[colorlinks=true, allcolors=blue]{hyperref}

\title{CRPL: Interim Report}
\author{Harrison Beau Barker}

\usepackage{graphicx,float}
\graphicspath{ {images/} }

%\usepackage{minted}
\usepackage{xcolor}

\usepackage{tabularx}
\newcolumntype{b}{X}
\newcolumntype{s}{>{\hsize=.5\hsize}X}

\newcommand{\keyword}[1]{\textbf{\textit{#1}}}
\newcommand{\q}[2]{\begin{quote} #1 \cite{#2} \end{quote}}

\setlength{\parskip}{1em}
%\setlength{\parindent}{1em}
\begin{document}

\begin{titlepage}
	\centering
	\includegraphics[width=0.4\textwidth]{crpl}\par
	\vspace{1cm}
	{\huge\bfseries CRPL: Report\par}
	\vspace{2cm}
	{\Large\itshape Harrison Beau Barker\par}
	\vfill
	{\url{https://github.com/MrHarrisonBarker/crpl}\par}
	\vspace{1cm}
	{\large Hbark002, 33575210\par}
\end{titlepage}

\abstract{TODO}

\tableofcontents{}

\section{Keywords}

\begin{description}
	\item[Copyright]
	\item[Blockchain]
	\item[Ethereum]
	\item[Smart Contract]
	\item[EVM] 
\end{description}

\section{Introduction}

The aim of this project is to represent a version of \keyword{copyright} for protection of intellectual property on a \keyword{blockchain} backed by a public ledger of transactions. My initial impetus for this project was a book I read in 2018 called \textit{"Blockchain Revolution: How the Technology Behind Bitcoin Is Changing Money, Business, and the World"}\cite{blockchain_revolution}, before this point I knew little of the applications for blockchain technologies attributing them only to a new form of digital currency allowing peer to peer transactions of wealth.

However, this book introduced the concept of \keyword{smart contracts} and the possibilities now small immutable programs could be saved and run on a blockchain. The most relevant change smart contracts brought was progressive state into an infamously immutable technology, by leveraging an unmodifiable chain of transactions state became not just a current record (like most traditional systems) but a historical account of all previous states with a clear a definable list of transformations precisely timestamped.

This new knowledge of \keyword{blockchains} came to fruition when it came to selecting my final project but to what problem should I help to provide a solution to leveraging this technology. A combination of recent interest in the crypto-sphere mainly coming from NFTs, continued displays of a \keyword{copyright} system not fit for purpose\cite{DMCA-abuse} and the book I had read 3 years earlier proposing that \keyword{blockchain} and \keyword{smart contracts} were an extremely viable solution to problems intersecting law and social structures.

\subsection{Unfit for purpose}

So why is the current \keyword{copyright} system not fit for purpose? I've broken it into three interlinking factors; complexity, jurisdiction dependence and lack of digital computerised systems. starting with complexity, it is often hard as a creator to know if your work is protected or if the protection is enough? This gives massive power to publishers, managers and companies who are willing to exploit this fact by blinding a creative with a large contract covered in legal jargon. Should an artist also have to be a lawyer?

Jurisdiction dependence can be looked at as a point of complexity and inefficiency in the system, yes there's international \keyword{copyright} law in the form of the \textit{"Berne Convention"}\cite{Berne} and later \textit{"WIPO Copyright Treaty"}\cite{WIPO} which informs most of our modern \keyword{copyright} law. However, these treaties only state minimum requirements and standards to follow but do not control the internal \keyword{copyright} law of a sovereign nation which will always take precedence.

This idea of global \keyword{copyright} links nicely into my last factor which is centred around the digital and ever more interconnected world we're living in. Even though in 1996 the \textbf{WCT}\cite{WIPO} was introduced to address issues caused by the emergence of information technology, however the world and more specifically the internet has changed an unimaginable amount since 1996 but \keyword{copyright} is effectively unchanged including the systems to register and view registered \keyword{copyrights} which are closed off in obscurity.

\subsection{What a solution needs}

I've defined what I believe as to be four requirements of a solution to this problem and how I've addressed these requirements.

% TODO not sure about this section
\begin{description}
	\item[Global] Both the system needed to be globally accessible, available and consistent across all jurisdictions to minimise complexity and maximise protection for an interconnected world. I've implemented this by using the \keyword{Ethereum} network which is made up of thousands of nodes across the world.
	\item[Open] Openness is essential to instil trust in a completely independent and alternative solution compared with government institutions that have implied trust and guaranty. I've implemented this by writing and using open-source code including \keyword{Ethereum} which is open-source and backed by a public ledger.
	\item[Robust] The current written laws and contracts maybe complex but are strongly defined with a commonly accepted interpretation, this will have to be true in this solution. I've implemented this by using an immutable \keyword{smart contract} for copyright representation and logic.
	\item[Simple] This is the most important requirement as the proposed problem is centred around current \keyword{copyright} complexity so any solution will need to be accessible to every possible user. I've implemented this by using clearly defined selectable \keyword{copyright} protections when registering a work.
\end{description}

\section{Research}
\subsection{Blockchain}
% TODO Technical knowledge needed to implement
% TODO How it works?
% TODO Why use it?

\subsection{Existing solutions}

\subsubsection{Copyright law}
% TODO The state of the current copyright system
\subsubsection{Online rights management}
% TODO How do creators manage protection of their work

\subsection{Development theory}
% TODO Explanation of agile development methodology
% TODO TDD and Scrum

\subsection{Aims of the solution}
% TODO The over arcing aims of the solution, aka what should the system fix

\section{Design}

\subsection{Technical Requirements}
% TODO a more comprehensive and detailed list of requirements

\begin{description}
	\item[Copyright smart contract] Immutable code on a public ledger “blockchain“ for the purpose of establishing ownership or the copyright to a piece of work.
	\item[Multi-party distribution] The ability to establish a complex ownership structure which includes multiple individuals/groups.
	\item[Ownership transfer] The ability to change the ownership of a copyright from one complex structure to another with consent of all current owner(s).
	\item[Work verification] Verification of a work to establish its originality with a reasonable accuracy for the platform.
	\item[Dispute filing] Allow any user to dispute a copyright with sufficient evidence and provide an option for resolving these disputes by the owner(s).
	\item[Digital signing] Digital signing a work for authentic verification based on our records.
%	\item[\color{orange}{Decentralised Work CDN & proxy}] Providing an access point for stored work within the chain.
%	\item[\color{orange}{Web-socket updates}] Real-time updates for the front-end UI to provide a better end-user experience.
\end{description}

\subsection{Smart contract}
% TODO Permission based protections

\subsubsection{Inspiration}

To design a smart contract without prior experience I decided look at what others had done before and because I knew my new contract was going to at least exhibit similar functionality and basic principles as \textbf{NFTs} I started with the EIP for non-fungible tokens \href{https://eips.ethereum.org/EIPS/eip-721}{EIP-721}. 

\begin{quote}
"Ethereum Improvement Proposals (EIPs) describe standards for the Ethereum platform, including core protocol specifications, client APIs, and contract standards."\cite{eip}
\end{quote}

This document describes a standard interface for all external methods and events an \textbf{NFT} contract should implement, most of these make sense straight away methods like: balanceOf, ownerOf, transferFrom and the Transfer event. Then there's a few methods to do with "approval" which is \keyword{Ethereum} language for access control, essentially what addresses are allowed to transact and make changes.

An interface spec is useful for understanding how the contract is supposed to interact with the outside world but nothing about how the contract works internally. So I went and found an implementation of this interface \href{https://github.com/OpenZeppelin/openzeppelin-contracts/blob/master/contracts/token/ERC721/ERC721.sol}{here} written by \href{https://github.com/OpenZeppelin}{OpenZeppelin} to gain an understanding of how these contracts operate.

\begin{verbatim}
	// Mapping from token ID to owner address
	mapping(uint256 => address) private _owners;

	// Mapping owner address to token count
	mapping(address => uint256) private _balances;

	// Mapping from token ID to approved address
	mapping(uint256 => address) private _tokenApprovals;

	// Mapping from owner to operator approvals
	mapping(address => mapping(address => bool)) private _operatorApprovals;
\end{verbatim}

This snippet is taken from the OpenZeppelin \href{https://github.com/OpenZeppelin/openzeppelin-contracts/blob/master/contracts/token/ERC721/ERC721.sol}{ERC721.sol} contract and is essentially how an \textbf{NFT} "works". a series of mappings that are saved in "storage" which is an area of the \keyword{EVM} that every smart contract has access to for storing state variables that need to persistent. These mappings are hashmaps from one type to another, first is the \_owners mapping which points to an owners address based on the hash of a given id (unit256). All the transactional methods are simply modifying these mappings, when you register a new token your wallet address is saved in the map entry for the next id.

\subsubsection{Ownership structure}

This core pattern/architecture was used as the foundation of my new contract, however there is one major requirement of my system not supported by the basic EIP-721 standard which is the ability of multiple address/people to have ownership of a token. This wasn't going to work for representing copyright as works can quite often involve multiple people collaborating the book I mentioned in the beginning of this report\cite{blockchain_revolution} has two authors, a system not representing the work and effort of all involved is just not acceptable.

\begin{figure}[H]
\caption{Structured Ownership essential mappings}
\centering
\includegraphics[width=0.5\textwidth,height=0.5\textheight,keepaspectratio]{images/operational/mappings.png}
\label{fig:float}
\end{figure}

To solve this problem I've redesigned how ownership is defined with the smart contract, instead of mapping the token id to one address the contract will now map to an \textbf{OwnershipStructure} which then points to a list of owner addresses along with a number of shares that specific address holds in the token.

This design obviously borrows a lot from limited companies share structure allowing for a complex ownership of multiple individuals or groups with implied variance in ownership (although the number of shares an address owns makes no immediate difference in the current implementation of this contract as this was outside of the desired complexity scope).

\subsubsection{Shareholder consensus}

Allowing multiple wallets to have ownership over a token now introduces a new problem for my contract design, when a change is made everyone has to agree to that change I can't just check if you're an owner anymore, giving the ability to change the copyright to everyone with a stake without consulting with all other owners is a point of exploitation.

\begin{figure}[H]
\caption{Structured Ownership proposal mappings}
\centering
\includegraphics[width=\textwidth,height=\textheight,keepaspectratio]{images/operational/prop-mappings.png}
\end{figure}

This is the solution I've designed for the shareholder consensus problem, now instead of making a direct change to the copyright (in this case an ownership restructure) a user proposes a change to the copyright which is then voted on by all the owners until a unanimous vote tally is reached then the change can be made.

\subsubsection{Permissions/Metadata}

% TODO is this section needed?

\subsection{Back-end}
% TODO Service and dependancy injection based architecture
% TODO Background service pattern
% TODO event processing patterns
% TODO state diagrams (how the flow of applications)

\subsection{Database}
% TODO what is duplicated between the chain and the database
% TODO what is only on the database and not on the chain
% TODO Express the need for synchronisation

\subsection{Front-end}
% TODO Visual design philosophy
% TODO How the use will login?
% TODO form design and working

\subsection{Architecture}
% TODO an overview of the system architecture design

\subsection{Development process}
% TODO how the development was laid out and planned
% TODO sprint structure

\section{Implementation}

\subsection{Smart contract}
% TODO saved metadata

\subsubsection{Interface overview}
% TODO events
% TODO ownership
% TODO copyright

\subsubsection{Registration}
% TODO Registration of copyright
% TODO how each copyright is saved (maximum size?)
% TODO approve the message sender
% TODO record all shareholders

\subsubsection{Ownership restructure}
% TODO Restructure proposal
% TODO Binding vote
% TODO Restructure event
% TODO failed restructure

\subsubsection{Modifiers}
% TODO isShareholder or approved
% TODO valid addresses
% TODO expired

\subsection{Back-end}
% TODO Service -> Controller
% TODO digital signing
% TODO ContractRepository

\subsubsection{Queries}
% TODO structured query
% TODO chain injection
\includegraphics[width=\textwidth,height=\textheight,keepaspectratio]{images/operational/chain-inject}

\subsubsection{Registration process}
\includegraphics[width=\textwidth,height=\textheight,keepaspectratio]{images/operational/cpy-registration-status-graph}

\subsubsection{Dispute handling}
% TODO filing disputes
% TODO resolving disputes: payment, change of ownership
\includegraphics[width=\textwidth,height=\textheight,keepaspectratio]{images/operational/dispute-workflow}

\subsubsection{Applications framework}
% TODO Data model structure (application, view model, input model)
% TODO unified status and processing flow (update -> submit = complete)

\subsubsection{API overview}
% TODO API diagrams and structure
\includegraphics[width=\textwidth,height=\textheight,keepaspectratio]{images/operational/Forms-Api}

\subsubsection{Blockchain event listeners}
% TODO Blockchain event listeners
\includegraphics[width=\textwidth,height=\textheight,keepaspectratio]{images/operational/Event-Listening}


\subsubsection{Background services}
% TODO verification pipeline
% TODO Expiry

\includegraphics[width=\textwidth,height=\textheight,keepaspectratio]{images/patterns/background-processing-pattern}

\subsubsection{Account management}
% TODO issues with transfer and delete account

\subsection{Database}
% TODO ChainSync™

\subsection{Front-end}

\section{Discussion}
\subsection{Limitation}
% TODO limits of the scope (having to reign in the scope of the project at all points)
% TODO limits of the implementation (gas/price aka no money involved at the moment, transfer and delete wallet, de-sync)
% TODO no way of licences for the use of a work

\subsection{Blockchain technology}
% TODO Are these NFTs?
% TODO How long will the blockchain?
% TODO Social consensus/ conflict with governments

\section{Evaluation}
\subsection{Process}
% TODO evaluating my development
% TODO was development agile?

\subsection{Product}
% TODO Has all the functional specifications been met
% TODO Has all the non-functional specs been met
% TODO Does the product fit the target users needs
% TODO Does the product fulfil

\section{Future development}

\begin{description}
	\item[Royalty payments]
	\item[External verification]
	\item[Analytics]
\end{description}

\section{Conclusion}


\bibliographystyle{plain}
\bibliography{sources.bib}

\end{document}
