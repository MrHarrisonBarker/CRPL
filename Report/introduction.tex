\section{Introduction}
\label{sec:intro}

The aim of this project is to represent a version of \keyword{copyright} for protection of intellectual property on a \keyword{blockchain} backed by a public ledger of transactions. My initial impetus for this project was a book I read in 2018 called \textit{"Blockchain Revolution: How the Technology Behind Bitcoin Is Changing Money, Business, and the World"}\cite{blockchain_revolution}, before this point I knew little of the applications for blockchain technologies attributing them only to a new form of digital currency allowing peer to peer transactions of wealth.

However, this book introduced the concept of \keyword{smart contracts} and the possibilities now small immutable programs could be saved and run on a blockchain. The most relevant change smart contracts brought was progressive state into an infamously immutable technology, by leveraging an unmodifiable chain of transactions state became not just a current record (like most traditional systems) but a historical account of all previous states with a clear a definable list of transformations precisely timestamped.

This new knowledge of \keyword{blockchains} came to fruition when it came to selecting my final project but to what problem should I help to provide a solution to leveraging this technology. A combination of recent interest in the crypto-sphere mainly coming from NFTs, continued displays of a \keyword{copyright} system not fit for purpose\cite{DMCA-abuse} and the book I had read 3 years earlier proposing that \keyword{blockchain} and \keyword{smart contracts} were an extremely viable solution to problems intersecting law and social structures.

\subsection{Unfit for purpose}
\label{sec:unfit}

So why is the current \keyword{copyright} system not fit for purpose? I've broken it into three interlinking factors; complexity, jurisdiction dependence and lack of digital computerised systems. starting with complexity, it is often hard as a creator to know if your work is protected or if the protection is enough? This gives massive power to publishers, managers and companies who are willing to exploit this fact by blinding a creative with a large contract covered in legal jargon. Should an artist also have to be a lawyer?

Jurisdiction dependence can be looked at as a point of complexity and inefficiency in the system, yes there's international \keyword{copyright} law in the form of the \textit{"Berne Convention"}\cite{Berne} and later \textit{"WIPO Copyright Treaty"}\cite{WIPO} which informs most of our modern \keyword{copyright} law. However, these treaties only state minimum requirements and standards to follow but do not control the internal \keyword{copyright} law of a sovereign nation which will always take precedence.

This idea of global \keyword{copyright} links nicely into my last factor which is centred around the digital and ever more interconnected world we're living in. Even though in 1996 the \textbf{WCT}\cite{WIPO} was introduced to address issues caused by the emergence of information technology, however the world and more specifically the internet has changed an unimaginable amount since 1996 but \keyword{copyright} is effectively unchanged including the systems to register and view registered \keyword{copyrights} which are closed off in obscurity.

\subsection{What a solution needs}

I've defined what I believe as to be four requirements of a solution to this problem and how I've addressed these requirements.

\begin{description}
	\item[Global] Both the system needed to be globally accessible, available and consistent across all jurisdictions to minimise complexity and maximise protection for an interconnected world. I've implemented this by using the \keyword{Ethereum} network which is made up of thousands of nodes across the world.
	\item[Open] Openness is essential to instil trust in a completely independent and alternative solution compared with government institutions that have implied trust and guaranty. I've implemented this by writing and using open-source code including \keyword{Ethereum} which is open-source and backed by a public ledger.
	\item[Robust] The current written laws and contracts maybe complex but are strongly defined with a commonly accepted interpretation, this will have to be true in this solution. I've implemented this by using an immutable \keyword{smart contract} for copyright representation and logic.
	\item[Simple] This is the most important requirement as the proposed problem is centred around current \keyword{copyright} complexity so any solution will need to be accessible to every possible user. I've implemented this by using clearly defined selectable \keyword{copyright} protections when registering a work.
\end{description}
